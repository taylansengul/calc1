\documentclass[../main.tex]{subfiles}

\begin{document}

If $f$ and $g$ are functions, then for every $x$ that belongs to the domains
of both $f$ and $g$ we define functions
\begin{itemize}
  \item $(f+g)(x)=f(x)+g(x)$
  \item $(f-g)(x)=f(x)-g(x)$
  \item $(fg)(x)=f(x)g(x)$
  \item $(f/g)(x)=f(x)/g(x)$ where $g(x)\neq 0.$
\end{itemize}

\begin{example}
Let $f(x)=\frac{1}{x+2}$ and $g(x)=\frac{x}{x-1}$. Find $(f+g)(x),(f-g)(x),$
$(fg)(x)=f(x)g(x)$ and $(f/g)(x)$ where $g(x)\neq 0.$
\end{example}

\subsection*{Composition of Functions}
If $f$ and $g$ are two functions, then
\[
  f\circ g(x)=f(g(x)).
\]
The domain of $f\circ g$ consists of those numbers $x$
in the domain of $g$ for which $g(x)$ is in the domain of $f$.


\begin{example}
Let $f(x)=\sqrt{x}$ and $g(x) = x+1$. Find $f\circ g$, $g \circ f$, $f \circ f$ and $g \circ g$. State the domains of each function.

\begin{table}[H]
  \centering
  \begin{tabular}{c c c}
    \hline
    Function & Formula & Domain \\
    \hline
    $f$ & $\sqrt{x}$ & $[0, \infty)$ \\
    $g$ & $x+1$ & $\mathbb{R}$ \\
    $f \circ g$ & $\sqrt{x+1}$ & $[-1, \infty)$ \\
    $g \circ f$ & $\sqrt{x}+1$ & $[0, \infty)$ \\
    $f \circ f$ & $x^{1/4}$ & $[0, \infty)$ \\
    $g \circ g$ & $x+2$ & $\mathbb{R}$ \\
    \hline
  \end{tabular}
\end{table}
\end{example}


\subsection*{Piecewise Defined Functions}
Functions such as
\[
g(x) =
\begin{cases}
    2x & \text{for } x<0 \\
    x^2 & \text{for }x\geq0
\end{cases}
\]
which are defined by different formulas on different intervals are sometimes called \textbf{piecewise defined functions.}

% \subsection*{Inverse Functions}
% Remember that a function is \textbf{one-to-one} if for every value in the range, there is exactly one value in the domain.

% A function is one-to-one if every horizontal line crosses its graph at most once, which is commonly known as the \textbf{horizontal line test}.
\end{document}