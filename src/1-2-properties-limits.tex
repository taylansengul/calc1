\documentclass[calc1-main.tex]{subfiles}

\begin{document}

The precise definition of the limit is not easy to use. Instead, there are a number of properties that
limits have which allow you to compute them without having to resort to the epsilon-delta definition.

The following statements remain true if one replaces each limit by a one-sided limit, or a limit for $x\to\infty$.

\begin{theorem}[Limits of constants and $x$]
  If $a$ and $c$ are constants, then
  \begin{equation}
  \lim_{x\to a}c=c \tag{$P_1$}
  \end{equation}
  and
  \begin{equation}
    \lim_{x\to a} x= a.\tag{$P_2$}
  \end{equation}
\end{theorem}
\begin{proof}
  We must find $\delta$. For ($P_2$) choose $\delta = \epsilon$. For ($P_1$) any $\delta$ will do.
\end{proof}

\begin{theorem}[Limits of sums, products, quotients]
  Suppose
  \[
  \lim_{x\to a}f(x)=L, \qquad \lim_{x\to a}g(x)=M.
  \]
  Then
  \begin{align}
    \lim_{x\to a}\bigl(f(x)+g(x)\bigr)=L+M, \tag{$P_3$} \\
    \lim_{x\to a}\bigl(f(x)-g(x)\bigr)= L - M, \tag{$P_4$} \\
    \lim_{x\to a}\bigl(f(x)\cdot g(x)\bigr)= L\cdot M \tag{$P_5$}
  \end{align}
  If $ \lim_{x\to a}g(x)\ne0$,
  \begin{equation}
    \lim_{x\to a}\frac{f(x)}{g(x)}= \frac{L}{M}.\tag{$P_6$}
  \end{equation}
  Finally, if $m$ and $n$ are integers such that $L^{m/n}$ is defined
  \begin{equation}
    \lim_{x\to a}(f(x))^{m/n}= L^{m/n}.\tag{$P_7$}
  \end{equation}
\end{theorem}

In other words the limit of the sum is the sum of the limits, etc.
One can prove these laws using the definition of limit.

\begin{proof}[Proof of $(P_3)$.]
    We can find $\delta_1 > 0$ such that $\abs{f(x) - L} < \frac\epsilon2$ and $\delta_2 > 0$ such that $\abs{g(x) - M} < \frac\epsilon2$.
    Now choose $\delta = \min\{\delta_1, \delta_2\}$.
    If $\abs{x-a} < \delta$, by triangle inequality
    \[
      \abs{f(x) + g(x) - L - M} \le \abs{f(x) - L} + \abs{g(x) - M} < \epsilon.
    \]
\end{proof}
\begin{proof}[Proof of $(P_5)$.]
  Let $\epsilon > 0$ and let
  \[
    \epsilon_1 = \min\{1, \frac{\epsilon}{1+|L|+|M|}\}
  \]
  Then $\epsilon_1>0$ and by definition, there exists $\delta_1>0$ and $\delta_2>0$ such that
  \[
    |f(x)-L|<\epsilon_1 \text{ whenever } 0 < |x-a|<\delta_1
  \]
  \[
    |g(x)-L|<\epsilon_1 \text{ whenever } 0 < |x-a|<\delta_2
  \]
  Let $\delta=\min\{\delta_1, \delta_2\}$.

  Use triangle inequality
  \[
    \begin{split}
      |f(x)g(x)-LM|
      & = |(f(x) - L + L) + (g(x) - M + M) -LM| \\
      & \le |f(x)-L||g(x)-M| + |M||f(x)-L| + |L||g(x)-M| \\
      & \le \epsilon_1^2 + |M|\epsilon_1 + |L| \epsilon_1 \\
      & \le \epsilon_1 + |M|\epsilon_1 + |L| \epsilon_1 \\
      & \le \epsilon_1 (1 + |M| + |L|) \le \epsilon
    \end{split}
  \]
\end{proof}

\begin{example}
  Find $\lim_{x\to2}x^2$.

  One has
  \begin{align*}
    \lim_{x\to2} x^2 &= \lim_{x\to2} x\cdot x \\
    &= \bigl( \lim_{x\to2} x\bigr)\cdot \bigl( \lim_{x\to2} x\bigr)
    &\text{by $(P_5)$}\\
    &= 2\cdot 2 = 4.
  \end{align*}
  Similarly,
  \begin{align*}
    \lim_{x\to2} x^3 &= \lim_{x\to2} x\cdot x^2 \\
    &= \bigl( \lim_{x\to2} x\bigr)\cdot \bigl( \lim_{x\to2} x^2\bigr)
    &\text{$(P_5)$ again}\\
    &= 2\cdot 4 = 8,
  \end{align*}
  and, by $(P_4)$
  \[
  \lim_{x\to2} x^2-1 = \lim_{x\to2} x^2 - \lim_{x\to2} 1 = 4-1 = 3,
  \]
  and, by $(P_4)$ again,
  \[
  \lim_{x\to2} x^3-1 = \lim_{x\to2} x^3 - \lim_{x\to2} 1 = 8-1 = 7,
  \]
  Putting all this together, one gets
  \[
  \lim_{x\to 2}\frac{x^3-1}{x^2-1} = \frac{2^3-1}{2^2-1} = \frac{8-1}{4-1}=
  \frac{7}{3}
  \]
  because of $(P_6)$.  To apply $(P_6)$ we must check that the denominator
  (``$L_2$'') is not zero.  Since the denominator is 3 everything is OK, and
  we were allowed to use $(P_6)$.
\end{example}
\end{document}
