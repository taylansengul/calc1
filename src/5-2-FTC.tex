\documentclass[calc1-main.tex]{subfiles}

\begin{document}
In this section we develop the relation between the integral and the derivative.

\subsection*{Antiderivative}
We will call $F(x)$ as an antiderivative of $f(x)$ if $F'(x) = f(x)$. For example $x$ is an antiderivative of $1$. Note that $x+1$ is also an antiderivative of $1$. So antiderivatives are not unique.

If $F$ and $G$ are antiderivatives of $f$ on an interval, so that $F'(x) = G'(x) = f(x)$ then
\[
  \frac{d}{dx} (F(x) - G(x)) = 0.
\]
But Theorem~\ref{THM: functions with zero derivative are constant} tells that $F(x) - G(x)$ must be a constant. Hence if $F(x)$ is an antiderivative of $f(x)$ then for any $C$, $F(x) + C$ is also an antiderivative of $f(x)$. Also any antiderivative of $f(x)$ is of  the form $F(x) + C$ for some $c$.

\begin{definition}
  The \textbf{indefinite integral} of $f(x)$ on interval $I$ is
  \[
    \int f(x) dx = F(x) + C
  \]
  provided $F'(x) = f(x)$ on $I$.
\end{definition}

\subsection*{The Fundamental Theorem of Calculus}
\begin{theorem}
~

  \textbf{PART I}. Suppose $f$ is continuous.
  \[
    \frac{d}{dx} \int_a^x f(t) dt = f(x)
  \]

  \textbf{PART II}. Suppose $f$ is differentiable.
  \[
    \int_a^b f'(x) dx = f(b) - f(a)
  \]
\end{theorem}

\begin{proof}
  For the first part, let
  \[
    F(x) = \int_a^x f(x) dx.
  \]
  Then
  \[
    F'(x) = \lim_{h \to 0} \frac{F(x+h) - F(x)}{h}
    = \lim_{h \to 0} \frac{1}{h} \left( \int_a^{x+h} f(t) dt - \int_a^{x} f(t) dt \right)
    = \lim_{h \to 0} \frac{1}{h} \int_x^{x+h} f(t) dt
  \]
Let $m(h)$ be the minimum, $M(h)$ be the maximum of $f$ on $[x, x+h]$. Then $m(h) \le f(t) \le M(h)$ on $x\le t \le x+h$. Thus
\[
  m(h) h = \int_x^{x+h} m(h) dx \le \int_x^{x+h} f(t) dt \le \int_x^{x+h} M(h) dx = M(h) h.
\]
Or
\[
  m(h) \le \frac{1}{h} \int_x^{x+h} f(t) dt \le M(h)
\]
Since $\lim_{h \to 0} m(h) = \lim_{h \to 0} M(h) = f(x)$, by Sandwich Theorem,
\[
  F'(x) = \lim_{h \to 0} \frac{1}{h} \int_x^{x+h} f(t) dt = f(x).
\]

Proof of the second part. Let
\[
  F(x) = \int_a^x f'(t) dt.
\]
Then by part I, $F'(x) = f'(x)$. We have seen that the only function whose derivative is zero on an interval is the constant function. Thus $F'(x) - f'(x) = 0$. Hence $F(x) - f(x) = c$, a constant. Since $0=F(a)$, $c= -f(a)$. And
\[
  \int_a^b f'(t) dt = F(b) = f(b) + c = f(b) - f(a).
\]
\end{proof}

Second part gives a method to evaluate definite integrals. To compute $\int_a^b f(x) dx$, find a function $F(x)$ whose derivative is $f(x)$. Then the value of $\int_a^b f(x) dx = F(b) - F(a)$.

We will use the evaluation symbol
\[
  F(x) \mid_a^b = F(b) - F(a).
\]
\begin{example}
  Evaluate
  \begin{enumerate}
    \item $\bigint_0^a x^2 dx = \frac{a^3}{3}$
    \item $\bigint_{-1}^2 (x^2-3x+2) dx = \frac{9}{2}$
  \end{enumerate}
\end{example}

\begin{example}
  Find the derivatives of the following functions.
  \begin{enumerate}
    \item $F(x) = \bigint_x^3 e^{-t^2} dt$
    \item $G(x) = \bigint_{-4}^{5x} e^{-t^2} dt$
    \item $H(x) = \bigint_{x^2}^{x^3} e^{-t^2} dt$
  \end{enumerate}
\end{example}
\begin{solution}
  By the Fundamental Theorem of Calculus Part I,
  \[
    F(x) = -\int_3^x e^{-t^2} \implies F'(x) = -e^{x^2}.
  \]
  Let $g(x) = \int_{-4}^x e^{-t^2} dt$. Then $G(x) = g(5x)$ and
  \[
    G'(x) = g'(5x) 5 = 5 e^{-(5x)^2}
  \]
  $H(x) = \int_{x^2}^a e^{-t^2} dt + \int_a^{x^3} e^{-t^2} dt$. Then
  \[
    H'(x) = e^{-x^6} 3x^2 - e^{x^4} 2x.
  \]
\end{solution}

In general
\[
  \frac{d}{dx} \int_{f(x)}^{g(x)} h(t) dt = h(g(x)) g'(x) - h(f(x)) f'(x).
\]

\rule{\textwidth}{1pt}
\begin{multicols}{2}
\begin{exercise}
~\\
  \begin{enumerate}
    \item $\displaystyle \frac{d}{dx} \int_2^x \frac{\sin t}{t} dt$

    Answer: $\frac{\sin x}{x}$

    \item $\displaystyle \frac{d}{dx} \int_{\sin x}^{\cos x} \frac{1}{1-x^2} dx$

    Answer: $-\frac{1}{\sin x} - \frac{1}{\cos x}$

    \item $\displaystyle \frac{d}{dx} F(\sqrt{x})$, if $F(t) = \int_0^t \cos(x^2)dx$

    Answer: $\frac{1}{2\sqrt{x}} \cos(x)$

    \item Find $H'(2)$ if $\displaystyle H(x) = 3x \int_4^{x^2} e^{\sqrt{t}}dt$

    Answer: $6e^2$

  \end{enumerate}
\end{exercise}
\end{multicols}
\rule{\textwidth}{1pt}

\end{document}
