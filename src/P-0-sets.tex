\documentclass[../main.tex]{subfiles}

\begin{document}
A set is a collection of elements.

$x\in A$ means $x$ is an element of the set $A$.
If $x$ is not a member of $A$, we write $x\notin A$.

$\varnothing$ is the set which contains no element and is called the empty set.

There are finite sets (ex. $\{0,1,2\}$) and infinite sets (ex. $\{0,1,2,3,...\}$).

If every element of the set $A$ is an element of the set $B$, we say that $A$ is subset of $B$, and write $A\subset B$.
\begin{example}
  List all the subsets of $\{0,1,2\}$.
\end{example}

For any set $A$, $A\subset A$ and $\varnothing \subset A$.

If $A\subset B$ and $B\subset A$, we write $A=B$.

$A\cap B = \{x:x\in A \text{ and } x\in B\}$ is called the intersection of $A$ and $B$.
\begin{center}
\begin{tikzpicture}[scale=1]
  \scope % A \cap B
  \clip (0,0) circle (1);
  \fill[gray, pattern=north west lines] (1,0) circle (1);
  \endscope
  % outline
  \draw (0,0) circle (1)
        (1,0) circle (1);
  \node (A) at (-1,1) {$A$};
  \node (B) at (2,1) {$B$};
  \node[red] (AB) at (0.5, 0) {$A \cap B$};
\end{tikzpicture}
\end{center}

If the intersection of two sets is the empty set, those sets are called disjoint.

$A\cup B=\{x:x\in A \text{ or } x\in B\}$ is called the union of $A$ and $B$.
\begin{center}
  \begin{tikzpicture}[scale=1]
    \fill[gray, pattern=north west lines] (1,0) circle (1);
    \fill[gray, pattern=north west lines] (0,0) circle (1);
    \draw (0,0) circle (1)
          (1,0) circle (1);
    \node at (-1,1) {$A$};
    \node at (2,1) {$B$};
    \node[red] at (0.5, 0) {$A \cup B$};
  \end{tikzpicture}
\end{center}

\begin{example}
For example if $A=\{0,1,2,5,8\}$ and $B=\{1,3,5,6\}$ then find $A\cap B$ and $A\cup B$.
\end{example}

The set of all elements in $A$ but not in $B$ is denoted $A\setminus B=\{x\in A:x\notin B\}$ and is called the complemet of $B$ in $A.$

\begin{example}
$\{0,2,3,5\}\setminus \{2,5,7,8\}=\{0,3\}$
\end{example}

$A\times B=\{(a,b):a\in A \text{ and } b\in B\}$ is called the Cartesian product of the sets $A$ and $B$.

\begin{example}
Write the cartesian product of $A=\{0,1,2\}$ and $B=\{2,3,4\}$.
\end{example}
\end{document}