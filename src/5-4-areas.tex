\documentclass[../calc1-main.tex]{subfiles}

\begin{document}
\begin{example}
  Find the area of the region lying above the x-axis and under the curve $y=3x-x^2$.
\end{example}
\begin{minipage}{0.5\textwidth}
  \begin{solution}
    The points where the graph intersects the x-axis are $y=0$ which gives $x=0$, $x=$.

    The area is
    \[
      \int_0^3 (3x-x^2) dx = \left. \left( \frac{3}{2}x^2 - \frac{1}{3}x^3 \right) \right\vert_0^3 = \frac{9}{2}.
    \]
  \end{solution}
\end{minipage}%
\begin{minipage}{0.5\textwidth}
  \begin{figure}[H]
    \centering
    \begin{tikzpicture}
    \begin{axis}[
    xmin=-1, xmax= 4, ymin=-1, ymax=3,
    xlabel=$x$, ylabel=$y$,
    xtick={0, 3},
    yticklabels={,,},
    axis lines=middle,
    width=8cm,]
    \addplot[domain=0:3, samples=100]  {3*x-x^2} node[pos=.5, above right] {$y=3x-x^2$};
    \addplot+[mark=none,
            domain=0:3,
            samples=100,
            pattern=north west lines,
            draw=black,
            pattern color=brown!50,
            area legend] {3*x-x^2} \closedcycle;
    \end{axis}
\end{tikzpicture}

  \end{figure}
\end{minipage}

\begin{example}
  Find the area of the region lying above the line $y=1$ and below the curve $y=\frac{5}{x^2+1}$.
\end{example}
\begin{minipage}{0.5\textwidth}
  \begin{solution}
    The curves $y=1$ and $y=\frac{5}{x^2+1}$ intersect at $x=\pm 2$. The area

    The area is
    \[
      \int_{-2}^2 \frac{5}{x^2+1} dx - \int_{-2}^2 1 dx = 5 \tan^{-1}x \mid_0^2 - 4 = 10\tan^.
    \]
  \end{solution}
\end{minipage}%
\begin{minipage}{0.5\textwidth}
  \begin{figure}[H]
    \centering
    \begin{tikzpicture}
    \begin{axis}[
    xmin=-4, xmax= 4, ymin=0, ymax=6,
    xlabel=$x$, ylabel=$y$,
    xtick={-2, 2},
    yticklabels={,,},
    title={$y=\frac{5}{x^2+1}$},
    axis lines=middle]
    \addplot[name path=F, domain=-3:3, samples=100]  {5/(x^2+1)};
    \addplot[name path=G, domain=-3:3]  {1};
    \addplot[pattern=north west lines, pattern color=brown!50] fill between[of=F and G, soft clip={domain=-2:2}];
    \node[below left] at (axis cs: 0,1) {$1$};
    \draw[thin, dashed] (axis cs: -2, 0) -- (axis cs: -2,1);
    \draw[thin, dashed] (axis cs: 2, 0) -- (axis cs: 2,1);
    \end{axis}
\end{tikzpicture}

  \end{figure}
\end{minipage}

Suppose $f(x) \le g(x)$ for $a \le x \le b$. Then the area of the region between these two curves and the lines $x=a$ and $x=b$ is
\[
	\text{Area} = \int_a^b (g(x) - f(x)) dx.
\]
\begin{example}
	Find the area of the bounded region lying between the curves $y=x^2 - 2x$ and $y= 4-x^2$.
\end{example}

\begin{minipage}{0.5\textwidth}
	\begin{solution}
		The two curves intersect at
		\[
			x^2-2x = 4-x^2 \implies 2x^2 - 2x - 4 = 0 \implies (x-2)(x+1) = 0
		\]
		So the intersection points are $x=-1$ and $x=2$.
		The area of the region is
		\[
			\text{Area} = \int_{-1}^2 \left(4-x^2) - (x^2 - 2x)  \right) dx = 9
		\]

	\end{solution}
\end{minipage}%
\begin{minipage}{0.5\textwidth}
  \begin{figure}[H]
	\centering
	\begin{tikzpicture}
\begin{axis}[axis lines=middle,
            xmin = -3, xmax = 4,
            xlabel=$x$,
            ylabel=$y$,
            enlargelimits,
            ytick=\empty,
            xtick={-1,2},
            width=8cm]
\addplot[name path=F, domain={-1.5:2.5}] {x^2-2*x} node[pos=.7, below]{$y=x^2-2x$};

\addplot[name path=G, domain={-1.5:2.5}] {4-x^2} node[pos=.4, right]{$y=4-x^2$};

\addplot[pattern=north west lines, pattern color=brown!50]fill between[of=F and G, soft clip={domain=-1:2}, reverse=true]
;

\draw[thin, dashed] (axis cs: -1, 0) -- (axis cs: -1,3);

\end{axis}
\end{tikzpicture}

\end{figure}
\end{minipage}

\begin{example}
	Find the area of the region bounded by $y=\sqrt{x}$ and $y=x^2$.
\end{example}
\begin{minipage}{0.5\textwidth}
  \begin{solution}
  	The curves intersect at
  	\[
  		\sqrt{x} = x^2 \implies x = x^4 \implies x(1-x^3) = 0.
  	\]
  	Hence the intersection points are $x=0$ and $x=1$.
  	\[
  		\text{Area} = \int_0^1 \left( \sqrt{x} - x^2 \right) dx = \frac{2}{3} - \frac{1}{3} = \frac{1}{3}.
  	\]
  \end{solution}
\end{minipage}%
\begin{minipage}{0.5\textwidth}
  \begin{figure}[H]
    	\centering
    	\begin{tikzpicture}
\begin{axis}[axis lines=middle,
            xmax = 2.5, ymax = 2,
            xlabel=$x$,
            ylabel=$y$,
            enlargelimits,
            ytick=\empty,
            xtick={0,1}]
\addplot[name path=F,domain={0:1.2}] {x^2} node[pos=1, above]{$y=x^2$};

\addplot[name path=G,domain={0:1.5}, samples=100] {sqrt(x)} node[pos=1, below]{$y=\sqrt{x}$};

\addplot[pattern=north west lines, pattern color=brown!50]fill between[of=F and G, soft clip={domain=0:1}];
\draw[thin, dashed] (axis cs: 1, 0) -- (axis cs: 1,1);

\end{axis}
\end{tikzpicture}

   \end{figure}
\end{minipage}

\begin{example}
	Find the area of the region lying to the right of the parabola $x=y^2-12$ and to the left of the straight line $y=x$.
\end{example}
\begin{minipage}{0.5\textwidth}
  \begin{solution}
  	The curves intersect at
  	\[
  		y^2 - 12 = y \implies (y-4)(y+3) = 0
  	\]
  	The intersection points are $y=4$ and $y=-3$.

  	\[
  		\text{Area} = \int_{-3}^4 \left( y - (y^2-12)  \right) dy = \frac{343}{6}
  	\]

  	An alternative way is to make the transformation $x \to y$, $y \to x$. The problem becomes finding the area between $y=x^2-12$ and $y=x$.
  \end{solution}
\end{minipage}%
\begin{minipage}{0.5\textwidth}
  \begin{figure}[H]
    	\centering
    	
\begin{tikzpicture}
    \begin{axis}[
            axis x line=middle,
            axis y line=middle,
            xlabel={$x$},
            ylabel={$y$},
            axis line style={<->},
            xmin=-13,xmax=5,
            ymin=-4,ymax=6,
            ytick={-3,4},
            samples=100,
        ]
        \addplot[name path=F,domain=-4:4]({x^2-12},{x});
        \addplot[name path=G,domain=-5:5] ({x},{x});
        \addplot[pattern=north west lines, pattern color=brown!50] fill between[of=F and G, soft clip={domain y=-3:4}, reverse=true];
    \end{axis}
\end{tikzpicture}

   \end{figure}
\end{minipage}

\rule{\textwidth}{1pt}
\begin{multicols}{2}
\begin{exercise}
~\\
  \begin{enumerate}
    \item Find the area bounded by the curves $y=x^2-2x$, and $y=3x-x^2$.

    Answer: $\frac{125}{24}$

    \item Find the area bounded by the curves $y=x^3$, and $y=x$ in the region $x \ge 0$.

    Answer: $\frac{1}{4}$.

    \item Find the area bounded by $y= \sin x$ and $y= \cos x$ between two consecutive intersections of these curves.

    Answer: $2\sqrt{2}$

    \item Find the area bounded by the curves $y=\frac{1}{x}$ and $2x+2y = 5$.

    Answer: $\frac{15}{8} - \ln4$

    \item Find the area bounded by the curves $x-y=7$ and $x=2y^2-y+3$.

    Answer: $9$
  \end{enumerate}
\end{exercise}
\end{multicols}
\end{document}
