\documentclass[calc1-main.tex]{subfiles}

\begin{document}
  Differentiability is stronger than continuity.
  \begin{theorem}
    If $f$ is differentiable at $x$ then $f$ is cts at $x$.
  \end{theorem}

  \begin{proof}
    \[
      \lim_{h \to 0} (f(x+h) - f(x)) =
      \lim_{h \to 0} \dfrac{f(x+h) - f(x)}{h}
      \lim_{h \to 0} h = f'(x) 0 = 0
    \]
    This means
    \[
      0 =
      \lim_{h \to 0} f(x+h) - \lim_{h \to 0} f(x) =
      \lim_{h \to 0} f(x+h) - f(x)
    \]
    Hence
    \[
      \lim_{h \to 0} f(x+h) = f(x)
    \]
  \end{proof}

  \begin{theorem}
    If $f$ and $g$ are differentiable at $x$ then
    \[
      (f+g)'(x) = f'(x) + g'(x),
    \]
    \[
      (f-g)'(x) = f'(x) - g'(x),
    \]
    and for any constant $c$
    \[
      (c f)'(x) = c f'(x).
    \]
  \end{theorem}
  \begin{proof}
    Let's prove the derivative of sums is sum of derivatives. The others are similar.
    \[
      \begin{split}
        (f+g)'(x) &=
        \lim_{h \to 0} \frac{(f+g)(x+h) - (f+g)(x)}{h} =
        \lim_{h \to 0} \frac{f(x+h)+g(x+h) - f(x)+g(x)}{h} \\
        & = \lim_{h \to 0} \frac{f(x+h)-f(x)}{h} +
        \lim_{h \to 0} \frac{g(x+h)-g(x)}{h} =
        f'(x) + g'(x),
      \end{split}
    \]
  \end{proof}

  The sum rule extends to any number of functions.
  \[
    (f_1 + \cdots + f_n)'(x) = f_1'(x) + \cdots + f_n'(x).
  \]

  \begin{example}
    Take the derivative of
    \[
      f(x) = 5 \sqrt{x} + \frac{3}{x} - 19
    \]
  \end{example}

  It is NOT true that derivative of product of functions is a product of their derivatives. Usually $(fg)'(x) \neq f(x) g(x)$.

  \begin{theorem}
    If $f$ and $g$ are differentiable at $x$ then
    \[
      (f g)'(x) = f'(x) g(x) + f(x) g'(x).
    \]
  \end{theorem}
  \begin{proof}
     \[
       \begin{split}
        (fg)'(x) = & \lim_{h \to 0} \frac{f(x+h)g(x+h)-f(x)g(x)}{h} =
        \lim_{h \to 0} \frac{\left( f(x+h)-f(x) \right)g(x+h) + f(x)\left( g(x+h) -g(x) \right)}{h} \\
        = & \lim_{h \to 0} \frac{f(x+h)-f(x)}{h} \lim_{h \to 0 }g(x+h) +
        \lim_{h \to 0} f(x) \lim_{h \to 0} \frac{g(x+h) -g(x) }{h}
       \end{split}
     \]
  \end{proof}

  \begin{example}
    Find the derivative of $f(x) = (x^2+x+1)(2x + \frac{1}{x})$.
  \end{example}

  The product rule can be extended to any number of functions
  \[
    (f_1 f_2 f_3)' = f_1' f_2 f_3 + f_1 f_2' f_3  + f_1 f_2 f_3'
  \]
  \[
    (f_1 \cdots f_n)' = f_1' f_2 \cdots f_3 + f_1 f_2' f_3 \cdots f_n + \cdots + f_1 \cdots f_{n-1} f_n'.
  \]

  \begin{theorem}
    If $f$ is differentiable at $x$ and $f(x) \neq 0$ then $1/f$ is diff at $x$, and
    \[
      \left(\dfrac{1}{f} \right)'(x) = \dfrac{-f'(x)}{f(x)^2}.
    \]
  \end{theorem}
  \begin{proof}
    \[
      \dfrac{d}{dx} \dfrac{1}{f(x)} = \lim_{h \to 0} \frac{\dfrac{1}{f(x+h)}-\dfrac{1}{f(x)}}{h} = \lim_{h \to 0} \dfrac{-1}{f(x+h) f(x)} \lim_{h \to 0} \dfrac{f(x+h)-f(x)}{h}
    \]
    The result follows by limit rules and continuity of $f$.
  \end{proof}

  \begin{example}
    Differentiate $y = \dfrac{x^5}{x^{2/3} + 1}$.
  \end{example}

  \begin{theorem}
    If $f$ and $g$ are differentiable at $x$ and $g(x) \neq 0$ then
    \[
      \left( \dfrac{f}{g} \right)'(x) =
      \dfrac{f'(x) g(x) - f(x) g'(x)}{g^2(x)}
    \]
  \end{theorem}
  \begin{proof}
    Using the product rule and reciprocal rule,
    \[
      \left( \dfrac{f}{g} \right)'(x) = \left( \dfrac{1}{g}(x) f(x) \right)' = \dfrac{f'(x) g(x) - f(x) g'(x)}{g^2(x)}
    \]
  \end{proof}

  \begin{example}
    Find the derivative of $f(x) = \dfrac{a + b x}{m + c x}$.
  \end{example}

  \begin{example}
    Find an equation of the tangent line to $y = \dfrac{2}{3-4 \sqrt{x}}$ at the point $(1, -2)$.
  \end{example}
  \begin{solution}
    Let us define $g(x)=3-4\sqrt{x}$. Then $g'(x)=-4 \frac{1}{2\sqrt{x}}=-\frac{2}{\sqrt{x}}$ and
    \[
      y'=2 \frac{-g'(x)}{g(x)^2}
      =2 \frac{\frac{2}{\sqrt{x}}}{(3-4\sqrt{x})^2}
      =\frac{4}{\sqrt{x}(3-4\sqrt{x})^2}
    \]
    Hence $y'(1)=4$. And the equation of the tangent line is $y=4(x-1)-2$.
  \end{solution}

  \begin{example}
    Find the x-coordinates of points on the curve $y=\frac{x+1}{x+2}$ where the tangent line is parallel to the line $y=4x$.
  \end{example}
  \begin{solution}
    Solving $y'=4$, we find $x=-3/2$ and $x=-5/2$.
  \end{solution}
  \begin{example}
    If $f(2)=2$ and $f'(2)=3$, calculate
    \[
      \left. \frac{d}{dx}\left( \frac{x^2}{f(x)} \right) \right\vert_{x=2}
    \]
  \end{example}
  \begin{solution}
    Answer is
    \[
      \frac{2\cdot2f(2)-2^2 f'(2)}{f(2)^2} = \frac{8-12}{4} = -1.
    \]
  \end{solution}
\end{document}
