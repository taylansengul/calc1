\documentclass[../calc1-main.tex]{subfiles}

\begin{document}

  The radian measure of an angle is defined to be the length of the arc of a unit circle corresponding to that angle.
  \[
    \text{angle in degrees = angle in radians}\cdot \frac{180^{\circ}}{\pi}.
  \]

  In calculus all angles are measured in radians. By an angle of $\pi/3$ we mean $\pi/3$ radians or $60^{\circ}$ not $(\pi/3)^{\circ} \approx 1.04^{\circ}$.

  \begin{theorem}
    $\lim_{\theta \to 0} \dfrac{\sin \theta}{\theta} = 1$.
  \end{theorem}
  \begin{proof}
    ~\\

    \begin{minipage}{0.5\textwidth}
      Suppose $0 < \theta < \frac{\pi}{2}$.

      Area of OQP triangle is $\frac{1}{2}\sin \theta \cos \theta$.

      Area of OAP arc is $\frac{\theta}{2\pi}\pi 1^2$.

      Area of OAT triangle is $\frac{1}{2} \tan \theta = \frac{\sin \theta}{2\cos \theta}$.
      \[
        \frac{1}{2}\sin \theta \cos \theta \le
        \frac{\theta}{2} \le
        \frac{\sin \theta}{2\cos \theta}
      \]
      Multiply by $\frac{2}{\sin \theta} > 0$
      \[
        \cos \theta \le
        \frac{\theta}{\sin \theta} \le
        \frac{1}{\cos \theta}
      \]
      Take reciprocal to get
      \begin{equation} \label{sin x over x inequality}
        \cos \theta \le
        \frac{\sin \theta}{\theta} \le
        \frac{1}{\cos \theta},
      \end{equation}
      for $0< \theta < \frac{\pi}{2}$.

      Use the squeeze theorem to show that
      \[
        \lim_{\theta \to 0+} \frac{\sin \theta}{\theta} = 1
      \]
      Similarly, we can show that \eqref{sin x over x inequality} holds for $-\frac{\pi}{2} < \theta < 0$ and hence
      \[
        \lim_{\theta \to 0-} \frac{\sin \theta}{\theta} = 1
      \]
    \end{minipage}
    \begin{minipage}{0.4\textwidth}
      \begin{figure}[H]
        \centering
        \pgfplotsset{soldot/.style={color=black,only marks,mark=*}}
\pgfplotsset{holdot/.style={color=black,fill=white,only marks,mark=*}}

\begin{tikzpicture}
  \begin{axis}[
  ymin=-1.2,
  ymax=1.5,
  xmin=-1.2,
  xmax=2.3,
  ticks=none,
  x=20mm,
  y=20mm,
  axis lines = middle,
  disabledatascaling
  ]
  \draw[black, thick] (axis cs:0,0) circle [radius=1];
  \draw[black] (axis cs:0.6,0) -- (axis cs: 0.6,0.8);
  \draw[black] (axis cs:0,0) -- (axis cs: 1,1.33);
  \draw[black] (axis cs:1,0) -- (axis cs: 0.6,0.8);
  \draw[black] (axis cs:1,0) -- (axis cs: 1,1.33);
  \addplot[soldot] coordinates{(0,0)(0.6,0)(0.6,0.8)(1,0)(1,1.33)};
  \node[below left] at (axis cs: 0,0) {$O$};
  \node[above right] at (axis cs: 1,0) {$A$};
  \node[above] at (axis cs: 1.2,0.8) {$P=(\cos \theta, \sin \theta)$};
  \node[below] at (axis cs: 0.6,0) {$Q$};
  \node[right] at (axis cs: 1,1.33) {$T=(1, \tan \theta)$};
  \node[right] at (axis cs: 0.8,0.6) {$\theta$};
  \draw [->] (axis cs:.2,0) arc [radius=.2,start angle=0,end angle=53.1];
  \node[right] at (axis cs: 0.15,0.15) {\tiny$\theta$};
  \node[above] at (axis cs: 0.25, 0.35) {\tiny 1};
\end{axis}
\end{tikzpicture}

      \end{figure}
    \end{minipage}

  \end{proof}

  \begin{example}
    Show that $\lim_{h \to 0} \frac{\cos h - 1}{h} = 0$.
  \end{example}
  \begin{solution}
    \[
      \begin{split}
      \lim_{h \to 0} \frac{\cos h - 1}{h} & =
      \lim_{h \to 0} \frac{(\cos h - 1)(\cos h + 1)}{h(\cos h + 1)} =
      \lim_{h \to 0} \frac{\cos^2 h - 1}{h(\cos h + 1)} \\
      &= \lim_{h \to 0} \frac{-\sin^2 h}{h(\cos h + 1)}
      = -\lim_{h \to 0} \frac{\sin h}{h} \frac{\sin h}{\cos h + 1} = -1 \cdot 0 = 0
      \end{split}
    \]
  \end{solution}
  \begin{theorem}
    $\sin x$ is differentiable for every $x$ and
    \[
      \frac{d}{dx} \sin x = \cos x
    \]
  \end{theorem}
  \begin{proof}
    \[
      \begin{split}
        \frac{d}{dx} \sin x & =
        \lim_{h \to 0} \frac{\sin(x+h) + \sin x}{h} =
        \lim_{h \to 0} \frac{\sin x \cos h + \cos x \sin h - \sin x}{h}\\ &=
        \lim_{h \to 0} \frac{\sin x (\cos h-1)}{h} + \lim_{h \to 0} \frac{\cos x \sin h}{h} =
        \sin x \lim_{h \to 0} \frac{(\cos h-1)}{h} + \cos x \lim_{h \to 0} \frac{\sin h}{h} = \cos x
      \end{split}
    \]
  \end{proof}

  \begin{theorem}
    $\cos x$ is differentiable for every $x$ and
    \[
      \frac{d}{dx} \cos x = -\sin x.
    \]
  \end{theorem}
  \begin{proof}
    \[
      \frac{d}{dx} \cos x =
      \frac{d}{dx} \sin\left(\frac{\pi}{2} -x\right) =
      - \cos\left(\frac{\pi}{2} -x\right) = - \sin x.
    \]
  \end{proof}
  \begin{example}
    Evaluate the derivative of
    \begin{itemize}
      \item[a)] $\sin(\pi x) + \cos(3x)$,
      \item[b)] $x^2 \cos(\sqrt{x})$,
      \item[c)] $\dfrac{\cos x}{1- \sin x}$
    \end{itemize}
  \end{example}

  \subsection*{The derivatives of the other trigonometric functions}
  \[
    \tan x = \frac{\sin x}{\cos x}, \quad
    \sec x = \frac{1}{\cos x}, \quad
    \cot x = \frac{\cos x}{\sin x}, \quad
     \csc x = \frac{1}{\sin x}.
  \]
  Since $\cos$ and $\sin$ are eveywhere differentiable, the above functions are differentiable everywhere except where their denominators are zero. The derivatives of these functions can be derived by using quotient and reciprocal rules.

  \[
    \frac{d}{dx} \tan x= \sec^2 x, \quad
    \frac{d}{dx} \sec x \tan x, \quad
    \frac{d}{dx} \cot x = -\csc^2 x, \quad
    \frac{d}{dx} \csc = -\csc x \cot x.
  \]

  \begin{example}
    Verify the derivative formulas for $\tan x$ and $\sec x$.
  \end{example}
  \begin{example}
    Find the derivative of $y = \sin(\cos(\tan t))$.
  \end{example}

  \begin{example}
    Find the points on the curve $y=\tan(2x)$, $-\pi/4 < x < \pi/4$, where the normal is parallel to the line $y=-x/8$.
  \end{example}


\end{document}
