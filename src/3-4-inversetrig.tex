\documentclass[calc1-main.tex]{subfiles}

\begin{document}

The six trigonometric functions are periodic and hence not 1-1. However we can restrict their domains in such a way that the restricted functions are 1-1.

\begin{minipage}{0.5\textwidth}
  The $\arcsin x$ or $\arcsin x$ is the inverse of the $\sin x$ restricted to $[-\pi/2, \pi/2]$,

  \begin{align*}
    & \sin (\arcsin y) = y, \qquad -1 \le y \le 1, \\
    & \arcsin (\sin x) = x, \qquad -\pi/2 \le x \le \pi/2.
  \end{align*}
\end{minipage}%
\begin{minipage}{0.5\textwidth}
  \begin{figure}[H]
    \centering
    \begin{tikzpicture}
  \begin{axis}[domain=-1:1, width=8cm, samples=500, axis lines*=middle, xtick={-1,1}, ytick={-1.57,1.57}, yticklabels={$-\pi$/2,$\pi$/2}]
    \addplot[thick]  {asin(x)/180*pi};
  \end{axis}
\end{tikzpicture}
    \caption{$f(x) = \arcsin x$ is a partial inverse of the sine function.}
  \end{figure}
\end{minipage}

\begin{example}
  Simplify
  \begin{enumerate}
    \item $\arcsin \frac{1}{2} = \frac{\pi}{6}$,
    % In[32]:= ArcSin[1/2] == Pi/6

    % Out[32]= True
    \item $\arcsin \frac{-\sqrt{2}}{2} = -\frac{\pi}{4}$,
    % In[33]:= ArcSin[-Sqrt[2]/2] == -Pi/4

    % Out[33]= True
    \item $\arcsin 2$ is undefined since $2$ is not in the range of sine.
  \end{enumerate}
\end{example}

\begin{example}
  Simplify
  \begin{enumerate}
    \item $\sin(\arcsin 0.7) = 0.7$,
    % In[34]:= Sin[ArcSin[0.7]] == 0.7

    % Out[34]= True
    \item $\arcsin (\sin 3 \pi/4) = \pi/4$,
    % In[35]:= ArcSin[Sin[3 Pi/4]] == Pi/4

    % Out[35]= True
    \item $\cos (\arcsin 0.6)$ = 0.8.
    % In[37]:= Cos[ArcSin[0.6]] == 0.8

    % Out[37]= True
    \begin{solution}
      Let $\theta=\arcsin 0.6$. By the Pythagorean Theorem, $\cos \theta = 0.8$.
    \end{solution}
    \item Similarly $\cos (\arcsin x) = \sqrt{1-x^2}$.
    % In[38]:= Cos[ArcSin[x]] == Sqrt[1 - x^2]

    % Out[38]= True
  \end{enumerate}
\end{example}

\begin{theorem}
\[
  \frac{d }{dx} \arcsin x = \frac{1}{\sqrt{1-x^2}}.
\]
% In[40]:= D[ArcSin[x], x] == 1/Sqrt[1 - x^2]

% Out[40]= True
\end{theorem}
\begin{proof}
  Let $y = \arcsin x$ so that $x = \sin y$. Then
  \[
    \frac{dy}{dx} = \frac{1}{\frac{dx}{dy}} = \frac{1}{\cos y} = \frac{1}{\sqrt{1-x^2}}.
  \]
\end{proof}

\subsection*{The Arctan Function}

Define the $y = \arctan x$ to be the inverse of $y=\tan x$ on $(-\pi/2, \pi/2)$.

\begin{align*}
  & \tan (\arctan x) = x, \qquad -\infty < x < \infty, \\
  & \arctan (\tan x) = x, \qquad -\pi/2 < x < \pi/2.
\end{align*}

\begin{figure}[H]
  \centering
  \begin{tikzpicture}
  \begin{axis}[domain=-5:5, samples=500, axis lines*=middle, xtick=\empty, ytick={-1.57,1.57}, yticklabels={$-\pi$/2,$\pi$/2}]
    \addplot[color = red]  {atan(x)/180*pi};
    \addplot[color = blue, dashed, domain=-5:0]  {-1.57};
    \addplot[color = blue, dashed, domain=0:5]  {1.57};
  \end{axis}
\end{tikzpicture}
  \caption{$f(x) = \arctan x$.}
\end{figure}

\begin{example}
  \begin{enumerate}
    \item $\tan (\arctan 3) = 3$,
    % In[31]:= Tan[ArcTan[3]] == 3

    % Out[31]= True
    \item $\arctan (\tan \frac{3\pi}{4}) = \arctan -1 = -\frac{\pi}{4}$
    % In[41]:= ArcTan[Tan[3 Pi/4]] == ArcTan[-1] == -Pi/4

    % Out[41]= True
    \item $\cos (\arctan x) = \frac{1}{\sqrt{1+x^2}}$
    % In[42]:= Cos[ArcTan[x]] == 1/Sqrt[1 + x^2]

    % Out[42]= True
  \end{enumerate}
\end{example}

\begin{theorem}
  $\frac{d \arctan(x)}{dx} = \frac{1}{1+x^2}$
\end{theorem}
% In[30]:= D[ArcTan[x], x] == 1/(1 + x^2) // Simplify

% Out[30]= True
\begin{proof}
  Let $y=\arctan x$ so that $x = \tan y$,
  \[
    \frac{d \arctan(x)}{dx} = \frac{1}{\frac{dx}{dy}} = \frac{1}{\sec^2 y} = \frac{1}{1+ \tan^2 y} = \frac{1}{1+x^2}.
  \]
\end{proof}


\begin{example}
  Find the slope of the curve $\arctan\left( \frac{2x}{y} \right) = \frac{\pi x}{y^2}$ at the point $(1, 2)$.
\end{example}
% In[27]:= eq = D[ArcTan[2 x/y[x]], x] == D[Pi x/y[x]^2, x];
% sol = Solve[eq //. {x -> 1, y[1] -> 2}, y'[1]];
% (y'[1] /. sol[[1]]) == (Pi - 2)/(Pi - 1)

% Out[29]= True
\begin{solution}
  Taking $\frac{d}{dx}$ of both sides
  \[
    \frac{1}{1+\left( \frac{2x}{y} \right)^2} 2\left( \frac{y-x y'}{y^2} \right) = \pi \left( \frac{y^2 - 2x y y'}{y^4} \right)
  \]
  Plugging $x=1$, $y=2$,
  \[
    2- y' = \pi (1- y') \implies y' = \frac{\pi-2}{\pi-1}.
  \]
\end{solution}

\subsection*{Other inverse trigonometric functions}
$\cos x$ is 1-1 on $[0, \pi]$ so we define $y=\arccos x$ as the inverse of $y=\cos x$ restricted to $[0, \pi]$.
\[
  y = \arccos x \iff x = \cos y \qquad 0 \le y \le \pi.
\]

\begin{theorem}
  \[
    \frac{d}{dx} \arccos x = -\frac{1}{\sqrt{1-x^2}}
  \]
  % In[44]:= D[ArcCos[x], x] == -1/Sqrt[1 - x^2]

  % Out[44]= True
\end{theorem}
For the derivative,
\[
  \frac{dy}{dx} = \frac{1}{\frac{dx}{dy}} = \frac{1}{-\sin y} = -\frac{1}{\sqrt{1-x^2}}
\]
Note that
\[
  \frac{d}{dx} \arccos x = -\frac{d}{dx} \arcsin x
\]
% In[45]:= D[ArcCos[x], x] == -D[ArcSin[x], x]

% Out[45]= True

The inverse and the derivative of other trigonometric functions can be defined similarly.

\subsection*{Quiz Problems}
\begin{example}
  Simplify
  \begin{enumerate}
    \item $\cos(\arctan \frac{1}{2}) = \frac{2}{\sqrt{5}}$
    % In[47]:= Cos[ArcTan[1/2]] == 2/Sqrt[5]

    % Out[47]= True
    \item $\tan (\arccos x)= \frac{\sqrt{1-x^2}}{x}$
    % In[49]:= Tan[ArcCos[x]] == Sqrt[1 - x^2]/x

    % Out[49]= True
  \end{enumerate}
\end{example}
\begin{example}
  Show that $\frac{d}{dx}(\arcsin x^2)^{1/2} = \frac{x}{\sqrt{1-x^4}\sqrt{\arcsin x^2}}$.
\end{example}
% In[53]:= D[Sqrt[ArcSin[x^2]], x] == x/(Sqrt[1 - x^4] Sqrt[ArcSin[x^2]])

% Out[53]= True
\end{document}
