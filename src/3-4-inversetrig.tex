\documentclass[../main.tex]{subfiles}

\begin{document}

The six trigonometric functions are periodic and hence not 1-1. However we can restrict their domains in such a way that the restricted functions are 1-1.

\begin{minipage}{0.5\textwidth}
  The $\sin^{-1}x$ or $\arcsin x$ is the inverse of the $\sin x$ restricted to $[-\pi/2, \pi/2]$,

  \begin{align*}
    & \sin (\sin^{-1} y) = y, \qquad -1 \le y \le 1, \\
    & \sin^{-1} (\sin x) = x, \qquad -\pi/2 \le x \le \pi/2.
  \end{align*}
\end{minipage}%
\begin{minipage}{0.5\textwidth}
  \begin{figure}[H]
    \centering
    \begin{tikzpicture}
  \begin{axis}[domain=-1:1, width=8cm, samples=500, axis lines*=middle, xtick={-1,1}, ytick={-1.57,1.57}, yticklabels={$-\pi$/2,$\pi$/2}]
    \addplot[thick]  {asin(x)/180*pi};
  \end{axis}
\end{tikzpicture}
    \caption{$f(x) = \arcsin x$.}
  \end{figure}
\end{minipage}

\begin{example}
  Simplify
  \begin{enumerate}
    \item $\sin^{-1} \frac{1}{2} = \frac{\pi}{6}$,
    \item $\sin^{-1} \frac{-\sqrt{2}}{2} = -\frac{\pi}{4}$,
    \item $\sin^{-1} 2$ is undefined since $2$ is not in the range of sine.
  \end{enumerate}
\end{example}

\begin{example}
  Simplify
  \begin{enumerate}
    \item $\sin(\sin^{-1} 0.7) = 0.7$,
    \item $\sin^{-1}(\sin 3 \pi/4) = \pi/4$,
    \item $\cos (\sin^{-1} 0.6)$.
    \begin{solution}
      Let $\theta=\sin^{-1} 0.6$. By the Pythagorean Theorem, $\cos \theta = 0.8$.
    \end{solution}
    \item Similarly $\cos (\sin^{-1} x) = \frac{1}{1-x^2}$.
  \end{enumerate}
\end{example}

Let $y = \sin^{-1} x$ so that $x = \sin y$. Then
\[
  \frac{dy}{dx} = \frac{1}{\frac{dx}{dy}} = \frac{1}{\cos y} = \frac{1}{\sqrt{1-x^2}}.
\]

Thus
\[
  \frac{d }{dx} \sin^{-1} x = \frac{1}{\sqrt{1-x^2}}.
\]

\subsection*{The Inverse Tangent (or Arctangent) Function}

Define the $\tan^{-1} x = \arctan x$ to be the inverse of $(-\pi/2, \pi/2)$.

\begin{align*}
  & \tan (\tan^{-1} x) = x, \qquad -\infty < x < \infty, \\
  & \tan^{-1} (\tan x) = x, \qquad -\pi/2 \le x \le \pi/2.
\end{align*}

\begin{figure}[H]
  \centering
  \begin{tikzpicture}
  \begin{axis}[domain=-5:5, samples=500, axis lines*=middle, xtick=\empty, ytick={-1.57,1.57}, yticklabels={$-\pi$/2,$\pi$/2}]
    \addplot[color = red]  {atan(x)/180*pi};
    \addplot[color = blue, dashed, domain=-5:0]  {-1.57};
    \addplot[color = blue, dashed, domain=0:5]  {1.57};
  \end{axis}
\end{tikzpicture}
  \caption{$f(x) = \arcsin x$.}
\end{figure}

\begin{example}
  \begin{enumerate}
    \item $\tan (\tan^{-1} 3) = 3$,
    \item $\tan^{-1} (\tan \frac{3\pi}{4}) = \tan^{-1} -1 = -\frac{\pi}{4}$
    \item $\cos (\tan^{-1} x) = \frac{1}{\sqrt{1+x^2}}$
  \end{enumerate}
\end{example}

Let $y=\tan^{-1} x$ so that $x = \tan y$,
\[
  \frac{dy}{dx} = \frac{1}{\frac{dx}{dy}} = \frac{1}{\sec^2 y} = \frac{1}{1+ \tan^2 y} = \frac{1}{1+x^2}.
\]

\begin{example}
  Find the slope of the curve $\tan^{-1}\left( \frac{2x}{y} \right) = \frac{\pi x}{y^2}$ at the point $(1, 2)$.
\end{example}
\begin{solution}
  Taking $\frac{d}{dx}$ of both sides
  \[
    \frac{1}{1+\left( \frac{2x}{y} \right)^2} 2\left( \frac{y-x y'}{y^2} \right) = \pi \left( \frac{y^2 - 2x y y'}{y^4} \right)
  \]
  Plugging $x=1$, $y=2$,
  \[
    2- y' = \pi (1- y') \implies y' = \frac{\pi-2}{\pi-1}.
  \]
\end{solution}
\subsection*{Other inverse trigonometric functions}
$\cos x$ is 1-1 on $[0, \pi]$ so we define $\cos^{-1} x$ for $\cos x$ restricted to $[0, \pi]$.
\[
  y = \cos^{-1} x \iff x = \cos y \qquad 0 \le y \le \pi.
\]

Note that $\sin(\cos^{-1} x) $
For the derivative,
\[
  \frac{dy}{dx} = \frac{1}{\frac{dx}{dy}} = \frac{1}{-\sin y} = -\frac{1}{\sqrt{1-x^2}}
\]
Note that
\[
  \frac{d}{dx} \cos^{-1}x = -\frac{d}{dx} \sin^{-1}x
\]

The inverse and the derivative of other trigonometric functions can be defined similarly.

\subsection*{Quiz Problems}
\begin{example}
  Simplify
  \begin{enumerate}
    \item $\cos(\tan^{-1} \frac{1}{2})$
    \item $\tan (\cos^{-1} x)$
  \end{enumerate}
\end{example}
\begin{example}
  Differentiate $(\sin^{-1} x^2)^{1/2}$.
\end{example}
\end{document}