\documentclass[calc1-main.tex]{subfiles}

\begin{document}

Consider $I = \displaystyle \int_a^b f(x) dx$, where $f$ is continuous on $(a, b)$.

If $a=-\infty$ or $b=\infty$ then we say $I$ is an \textbf{improper integral of type I.}

If $f$ is unbounded as $x$ approaches to $a$ or $b$ then we say $I$ is an \textbf{improper integral of type II.}

\subsection*{Improper Integrals of Type I}
\begin{definition}
  If $f$ is continuous on $[a, \infty)$
  \[
    \int_a^{\infty} f(x) dx = \lim_{R \to \infty} \int_a^R f(x) dx
  \]
  Similarly if $f$ is continuous on $(-\infty, b]$
  \[
    \int_{-\infty}^b f(x) dx = \lim_{R \to -\infty} \int_R^b f(x) dx
  \]
  In either cases, if the limit is finite, we say the integral \textbf{converges} and if the limit does not exists, we say the integral \textbf{diverges}. If the limit is $\infty$ or $-\infty$ we say the integral diverges to $\infty$ or $-\infty$.
\end{definition}

\begin{example}
	Find the area of the region lying under the curve $y=\frac{1}{x^2}$ and above the x-axis to the right of $x=1$.
\end{example}

\begin{minipage}{0.5\textwidth}
  \begin{solution}
  	The area is $A = \displaystyle \int_1^{\infty} \frac{dx}{x^2}$ which is an improper integral  of type-I.
  	\[
  	\begin{split}
  		A = &
  		\int_1^{\infty} \frac{dx}{x^2} = \lim_{R \to \infty} \int_1^R \frac{dx}{x^2} \\
  		= & \lim_{R \to \infty} \left.\left( -\frac{1}{x} \right)\right \vert_1^R=
  		\lim_{R \to \infty} \left( -\frac{1}{R} + 1 \right) =
  		1
  	\end{split}
  	\]
  \end{solution}
\end{minipage}%
\begin{minipage}{0.5\textwidth}
	\begin{figure}[H]
  \centering
  \begin{tikzpicture}
    \begin{axis}[
    xmin=.5, xmax= 3, ymin=0, ymax=1.6,
    % xlabel=$x$, ylabel=$y$,
    xtick={1},
    yticklabels={,,},
    axis lines=middle,
    width=6cm,]
    \addplot[domain=.7:3, samples=100]  {1/(x^2)} node[pos=.5, above right] {$y=1/x^2$};
    \addplot+[mark=none,
            domain=1:3,
            samples=100,
            pattern=north west lines,
            draw=black,
            pattern color=brown!50,
            area legend] {1/(x^2)} \closedcycle;
    \end{axis}
	\end{tikzpicture}
	\end{figure}
\end{minipage}



\begin{example}
	Find the area of the region lying under the curve $y=\frac{1}{x}$ and above the x-axis to the right of $x=1$.
\end{example}
\begin{minipage}{0.5\textwidth}
  \begin{solution}
  	The area is
  	\[
  	\begin{split}
  		A & = \int_1^{\infty} \frac{dx}{x} =
  		\lim_{R \to \infty} \int_1^R \frac{dx}{x} \\
  		& = \lim_{R \to \infty} \left. \ln x \right \vert_1^R=
  		\lim_{R \to \infty} \ln R =
  		\infty
  	\end{split}
  	\]
  \end{solution}
\end{minipage}%
\begin{minipage}{0.5\textwidth}
\begin{figure}[H]
  	\centering
  \begin{tikzpicture}
    \begin{axis}[
    xmin=.5, xmax= 3, ymin=0, ymax=1.5,
    % xlabel=$x$, ylabel=$y$,
    xtick={1},
    yticklabels={,,},
    axis lines=middle,
    width=6cm,]
    \addplot[domain=.7:3, samples=100]  {1/x} node[pos=.5, above right] {$y=1/x$};
    \addplot+[mark=none,
            domain=1:3,
            samples=100,
            pattern=north west lines,
            draw=black,
            pattern color=brown!50,
            area legend] {1/x} \closedcycle;
    \end{axis}
\end{tikzpicture}
\end{figure}
\end{minipage}

\begin{definition}
  For integrals of the form $\displaystyle \int_{-\infty}^{\infty} f(x) dx$, we define
  \[
    \int_{-\infty}^{\infty} f(x) dx = \int_{-\infty}^0 f(x) dx + \int_0^{\infty} f(x) dx
  \]
  The integral on the left converges if and only if both integrals on the right converges.
\end{definition}

Note that $\int_{-\infty}^{\infty} f(x) dx$ and $\lim_{R \to \infty} \int_{-R}^{R} f(x) dx$ may or may not be equal. For example $\int_{-\infty}^{\infty} x dx$ is divergent according to our definition since $\int_{0}^{\infty} x dx = \infty$ is divergent. But $\lim_{R \to \infty} \int_{-R}^{R} x dx = 0$.

\begin{example}
	Evaluate $\displaystyle \int_{-\infty}^{\infty} \frac{1}{1+x^2} dx$.
\end{example}
\begin{minipage}{0.5\textwidth}
\begin{solution}
	\[
		I =
		\int_{-\infty}^{\infty} \frac{1}{1+x^2} dx =
		\int_{-\infty}^0 \frac{1}{1+x^2} dx + \int_0^{\infty} \frac{1}{1+x^2} dx = 2 \int_0^{\infty} \frac{1}{1+x^2} dx
	\]
	since the integrand is an even function.
	\[
		\int_0^{\infty} \frac{1}{1+x^2} dx =
		\lim_{R \to \infty}\int_0^R \frac{dx}{1+x^2} =
		\lim_{R \to \infty}\int_0^R \arctan R =
		\frac{\pi}{2}
	\]
	So the answer is $I = \pi$.
\end{solution}
\end{minipage}%
\begin{minipage}{0.5\textwidth}
  \begin{figure}[H]
  	\centering
  	\begin{tikzpicture}
  	    \begin{axis}[
  	    xmin=-2, xmax= 2, ymin=0, ymax=1.5,
  	    % xlabel=$x$, ylabel=$y$,
  	    xtick={0},
  	    yticklabels={,,},
  	    axis lines=middle,
  	    width=6cm,]
  	    \addplot[domain=-2:2, samples=100]  {1/(1+x^2)} node[pos=.5, above] {$y=\frac{1}{1+x^2}$};
  	    \addplot+[mark=none,
  	            domain=-2:2,
  	            samples=100,
  	            pattern=north west lines,
  	            draw=black,
  	            pattern color=brown!50,
  	            area legend] {1/(1+x^2)} \closedcycle;
  	    \end{axis}
  	\end{tikzpicture}
  \end{figure}
\end{minipage}


\subsection*{Improper Integrals of Type-II}
\begin{definition}
	If $f$ is continuous on the interval $(a,b]$ and is possibly unbounded near $a$ then
	\[
		\int_a^b f(x) dx = \lim_{c \to a+} \int_c^b f(x) dx
	\]
	Similarly, if $f$ is continuous on the interval $[a,b)$ and is possibly unbounded near $b$ then
	\[
		\int_a^b f(x) dx \lim_{c \to b-} \int_a^c f(x) dx
	\]
\end{definition}

\begin{example}
	Find the area of the region lying under $y=1/\sqrt{x}$, above the $x$-axis, between $x=0$ and $x=1$.
\end{example}

\begin{minipage}{0.5\textwidth}
\begin{solution}
The area is
	\[
		\int_0^1 \frac{1}{\sqrt{x}} dx =
		\lim_{c \to 0+} \int_c^1 \frac{1}{\sqrt{x}} dx =
		\lim_{c \to 0+} \left. 2 x^{1/2} \right \vert_c^1 =
		\lim_{c \to 0+} (2 - 2\sqrt{c}) = 2
	\]
\end{solution}
\end{minipage}%
\begin{minipage}{0.5\textwidth}
  \begin{figure}[H]
  	\centering
  	\begin{tikzpicture}
  	    \begin{axis}[
  	    xmin=0, xmax= 1.5, ymin=0, ymax=2,
  	    % xlabel=$x$, ylabel=$y$,
  	    xtick={1},
  	    yticklabels={,,},
  	    axis lines=left,
  	    width=6cm,]
  	    \addplot[domain=.1:1.5, samples=100]  {1/sqrt(x)} node[pos=.9, above] {$y=\frac{1}{\sqrt{x}}$};
  	    \addplot+[mark=none,
  	            domain=0:1,
  	            samples=100,
  	            pattern=north west lines,
  	            draw=black,
  	            pattern color=brown!50,
  	            area legend] {1/sqrt(x)} \closedcycle;
  	    \end{axis}
  	\end{tikzpicture}
  \end{figure}
\end{minipage}



\begin{example}
	Evaluate $\displaystyle \int_0^2 \frac{dx}{\sqrt{2x-x^2}}$
\end{example}

\begin{minipage}{0.5\textwidth}
\begin{solution}
By the substitution $u = x-1$
\[
	I = \int_0^2 \frac{dx}{\sqrt{2x-x^2}} =
	\int_0^2 \frac{dx}{\sqrt{1-(x-1)^2}} =
	\int_{-1}^1 \frac{du}{\sqrt{1-u^2}}
\]
By the even symmetry,
\[
	I = 2\int_0^1 \frac{du}{\sqrt{1-u^2}}
\]
This is an improper integral of Type-II as the integrand is unbounded at $u=1$.
\[
	I = 2 \lim_{c \to 1-} \int_0^c \frac{du}{\sqrt{1-u^2}} =
	2 \lim_{c \to 1-} \left. \arcsin u \right \vert_0^c =
	2 \lim_{c \to 1-} \arcsin c = 2 \arcsin 1 = \pi
\]
\end{solution}
\end{minipage}%
\begin{minipage}{0.5\textwidth}
  \begin{figure}[H]
  	\centering
  	\begin{tikzpicture}
  	    \begin{axis}[
  	    xmin=0, xmax= 2.5, ymin=0, ymax=2,
  	    % xlabel=$x$, ylabel=$y$,
  	    xtick={1},
  	    yticklabels={,,},
  	    axis lines=left,
  	    width=6cm,]
  	    \addplot[domain=0:1.9, samples=100]  {1/sqrt(2*x-x^2)} node[pos=.6, right] {$y=\frac{1}{\sqrt{2x-x^2}}$};
  	    \addplot+[mark=none,
  	            domain=0:1.9,
  	            samples=100,
  	            pattern=north west lines,
  	            draw=black,
  	            pattern color=brown!50,
  	            area legend] {1/sqrt(2*x-x^2)} \closedcycle;
  	    \end{axis}
  	\end{tikzpicture}
  \end{figure}
\end{minipage}



\rule{\textwidth}{1pt}
\begin{multicols}{2}
\begin{exercise}
Evaluate the following integrals or show they diverge.
	\begin{enumerate}
		\item $\displaystyle \int_0^{\infty} \cos x dx$

		Answer: the integral diverges.
		\item $\displaystyle \int_0^{\infty} e^{-2x} dx$

		Answer: $1/2$.
		\item $\displaystyle \int_0^{\infty} x e^{-x} dx$

		Answer: $1$.
		\item $\displaystyle \int_0^{\pi/2} \tan x dx$

		Answer: the integral diverges to $\infty$.
		\item $\displaystyle \int_0^1 \frac{dx}{x}$

		Answer: the integral diverges to $\infty$.
		\item $\displaystyle \int_0^1 \ln x dx$

		Answer: $-1$.

		\item $\displaystyle \int_{-\infty}^{\infty} x e^{-x^2} dx$

		Answer: $0$.
	\end{enumerate}
\end{exercise}
\end{multicols}
\rule{\textwidth}{1pt}

\end{document}
