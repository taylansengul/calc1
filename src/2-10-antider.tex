\documentclass[calc1-main.tex]{subfiles}

\begin{document}

An \textbf{antiderivative} of a function on an interval $I$ is another function $F$ satisfying
\[
    F'(x) = f(x).
\]

\begin{example}
    Find an antiderivative of
    \begin{itemize}
        \item $f(x) = 1$
        \item $f(x) = \cos x$
    \end{itemize}
\end{example}

\begin{itemize}
    \item Antiderivatives of continuous functions always exist. We'll see later that $\int_0^x f(x) dx$ is an antiderivative of $f(x)$.
    \item Antiderivatives are not unique. If $C$ is any constant then $F(x) = x + C$ is an antiderivative of $f(x) = 1$.
    \item Two antiderivatives of a function differ by a constant. If $F$ and $G$ are two antiderivatives of $f$ then
    \[
        \frac{d}{dx} (F(x) - G(x)) = 0.
    \]
    By one of the consequences of the Mean Value Theorem, which says that if the derivative of a function is zero in an interval, that function must be constant in that interval. So $F(x) = G(x) + C$ for some constant $C$.
\end{itemize}


The \textbf{indefinite integral} of $f(x)$ is
\[
    \int f(x) dx = F(x) + C
\]
provided $F'(x) = f(x)$.

The symbol $\int$ is called the \textbf{integral sign}. The constant $C$ is called a \textbf{constant of integration}.
\begin{example}
    \[
        \int x dx = \frac{1}{2}x^2 + C
    \]
    \[
        \int (x^3 - 5x^2 +7) dx = \frac{1}{4}x^4 - \frac{5}{3}x^3 + 7x + C
    \]
\end{example}

\begin{example}
    Find the function $f(x)$ whose derivative is $f'(x) = 6 x^2 - 1$ for all $x$ and $f(2) = 10$.
\end{example}
\begin{solution}
    \[
        f(x) = \int (6x^2 - 1) dx = 2x^3 - x + C
    \]
    Use the condition to find $C = -4$.
\end{solution}
\end{document}
