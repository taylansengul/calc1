\documentclass[../main.tex]{subfiles}

\begin{document}

\begin{definition}[Right Limit]
  We say that
  \begin{equation}\label{eq:one-sided-lim-formulation}
    \lim_{x\searrow a} f(x) = L
  \end{equation}
  if for every $\epsilon>0$ one can find a $\delta>0$ such that
  \[
  a<x<a+\delta \implies |f(x)-L|<\epsilon.
  \]
  This is called right-limit of $f$ at $x=a$.
\end{definition}

The left-limit, i.e. the one-sided limit in which $x$ approaches $a$
through values less than $a$ is defined in a similar way.

\begin{theorem}
If both one-sided limits
\[
\lim_{x\searrow a} f(x) = L_{+}, \text{ and } \lim_{x\nearrow a} f(x) =
L_{-}
\]
exist, then
\[
\lim_{x\to a} f(x) \text{ exists} \iff L_{+}=L_{-}.
\]
\end{theorem}

In other words, if a function has both left- and right-limits at some
$x=a$, then that function has a limit at $x=a$ if the left- and
right-limits are equal.

\begin{example}
    Show that $\Lim{x}{0^+} \sqrt{x} = 0$.

    Solution: Choose $\delta = \epsilon^2$.
\end{example}

What happens to $f(x)$ as $x$ becomes ``larger and larger'' and ask?
\begin{definition}[Limit at Infinity]
  Let $f$ be some function which is defined on some interval $x_0<x<\infty$.
  If there is a number $L$ such that for every $\epsilon>0$ one can find
  an $A$ such that
  \[
  x>A \implies |f(x) - L| <\epsilon
  \]
  for all $x$, then we say that the limit of $f(x)$ for $x\to\infty$ is $L$.
\end{definition}

\begin{example}
  To \emph{prove} that $\lim_{x\to\infty}1/x = 0$ we apply the definition to
  $f(x) = 1/x$, $L=0$.

  For given $\epsilon>0$ we need to show that
  \begin{equation}\label{eq:1overx-small-for-x-large}
    \left|\frac1x - L\right|<\epsilon \text{ for all } x>A
  \end{equation}
  provided we choose the right $A$.

  How do we choose $A$?  $A$ is not allowed to depend on $x$, but it may
  depend on $\epsilon$.

  If we assume for now that we will only consider positive values of $x$,
  then \eqref{eq:1overx-small-for-x-large} simplifies to
  \[
  \frac 1x<\epsilon
  \]
  which is equivalent to
  \[
  x>\frac1\epsilon.
  \]
  This tells us how to choose $A$.  Given any positive $\epsilon$, we will
  simply choose
  \[
  A=\frac1\epsilon
  \]
  Then one has $|\frac1x-0| = \frac1x <\epsilon$ for all $x>A$.  Hence we
  have proved that $\lim_{x\to\infty}1/x=0$.
\end{example}

\subsection*{Infinite Limits}
\begin{definition}
    $\Lim{x}{a} f(x) = \infty$ if for every $B>0$ there exists $\delta>0$ such that $f(x) > B$ whenever $0<\abs{x-a}<\delta$.
\end{definition}

\begin{example}
    Verify that $\Lim{x}{0} \frac{1}{x^2} = \infty$.

    Solution: Choose $\delta = 1/\sqrt{B}$.
\end{example}
\end{document}