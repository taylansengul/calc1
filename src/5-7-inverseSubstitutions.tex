\documentclass[calc1-main.tex]{subfiles}

\begin{document}

\subsection*{The Inverse Sine Substitution}
If an integral involves $\sqrt{a^2-x^2}$, try the substitution $x = a \sin \theta$ or $\theta = \sin^{-1}\frac{x}{a}$.

We can assume $a>0$. Notice that $\sqrt{a^2-x^2}$ makes sense only when $-a \le x \le a$ which corresponds to $-\pi/2 \le \theta \le \pi/2$ so that $\cos \theta \ge 0$. Hence
\[
	\sqrt{a^2-x^2} = \sqrt{a^2(1-\sin^2\theta)} = a \sqrt{\cos^2 \theta}= a \abs{\cos \theta} = a \cos \theta.
\]

\begin{example}
	Evaluate $I = \displaystyle \int \frac{dx}{(5-x^2)^{3/2}}$.
\end{example}
\begin{solution}
	Let $x=\sqrt{5}\sin\theta$, $dx = \sqrt{5} \cos \theta d\theta$.
	\[
		(5-x^2)^{3/2} = (5 - 5 \sin^2\theta)^{3/2} = 5^{3/2} \abs{\cos\theta}^3 = 5^{3/2} \cos^3 \theta.
	\]
	since $\cos \theta \ge 0$.
	So
	\[
		I = \int \frac{\sqrt{5} \cos \theta d\theta}{5^{3/2} \cos^3\theta} = \frac{1}{5}\int \sec^2\theta d\theta = \frac{1}{5} \tan \theta + C = \frac{1}{5} \frac{x}{\sqrt{5-x^2}} + C
	\]
	The last equality can be found using $\sin \theta = \frac{x}{\sqrt{5}}$.
\end{solution}
\subsection*{The inverse Tangent Substitution}
If an integral involves $\sqrt{a^2+x^2}$ or $\dfrac{1}{x^2 + a^2}$, try the substitution $x = a \tan \theta$ or $ \theta = \tan^{-1} \frac{x}{a}$.

Since $x$ can take any real value, we have $-\pi/2 < \theta < \pi/2$ so that $\sec \theta > 0$. Assuming $a>0$,
\[
	\sqrt{a^2+x^2} = \sqrt{a^2(1+\tan^2\theta)} = a \sqrt{\sec^2 \theta}= a \abs{\sec \theta} = a \sec \theta.
\]
\begin{example}
	Evaluate $I = \displaystyle \int \frac{dx}{\sqrt{4+x^2}}$.
\end{example}
\begin{solution}
	Let $x= 2 \tan \theta$, $dx = 2 \sec^2 \theta d\theta$.
	\[
		\sqrt{4+x^2} = 2 \sqrt{\sec^2 \theta} = 2 \abs{\sec \theta} = 2 \sec \theta
	\]
	since $\sec \theta > 0$. Using $\tan \theta = x/2$ we can find $\sec \theta = \frac{\sqrt{4+x^2}}{2}$ and
	\[
		I = \int \sec \theta d\theta = \ln \abs{\sec \theta + \tan \theta} + C = \ln \abs{\frac{\sqrt{4+x^2}}{2} + \frac{x}{2}} + C
	\]
\end{solution}

You are not responsible for the inverse secant transformation which can be used to solve integrals involving $\sqrt{x^2-a^2}$.

\rule{\textwidth}{1pt}
\begin{multicols}{2}
\begin{exercise}
~\\
	\begin{enumerate}
		\item $\bigint \frac{x^2}{\sqrt{1-4x^2}} dx$,

		Answer: $-\frac{1}{8}x\sqrt{1-4x^2} + \frac{1}{16} \arcsin(2x) + C$

		\item $\bigint \frac{1}{x \sqrt{9-x^2}} dx$

		Answer: $\frac{1}{3}\ln\abs{x} - \frac{1}{3} \abs{3+\sqrt{9-x^2}} + C$.

		\item $\bigint \frac{1}{x^2 + 2x + 10} dx$

		Answer: $\frac{1}{3} \arctan\left( \frac{1+x}{3} \right) + C$
	\end{enumerate}
\end{exercise}
\end{multicols}
\rule{\textwidth}{1pt}

\end{document}
