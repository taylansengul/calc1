\documentclass[../calc1-main.tex]{subfiles}

\begin{document}

The \textbf{integers} are $\mathbb{Z}= \left\{ \dots, -2, -1, 0,1, 2, \dots \right\}$.
The \textbf{rational numbers} are $\mathbb{Q=}\left\{ \frac{m}{n}:m,n\in \mathbb{Z}\text{ and }n\neq 0\right\}$.

\textit{Pythagoreans thought that all numbers are ratios of integers. The discovery of irrational numbers is said to have shocked them.}

\begin{example}
Prove that $\sqrt{2}$ is not a rational number.
\end{example}
\begin{solution}
  Suppose that it is rational. Then $\sqrt{2}=m/n$, where $m,n\in \mathbb{Z}$ \ and $n\neq 0$. Also assume $m$ and $n$ have no common divisor.
  \[
    m^{2}/n^{2}=2  \implies m^{2}=2n^{2}
  \]
  Thus $m$ is even and we can write $m=2k$, where $k\in \mathbb{Z}$.
  \[
    4k^{2}=2n^{2} \implies n^{2}=2k^{2}
  \]
  Thus n is also even. But m and n cannot both be even. Accordingly, there can be no rational number whose square is 2.
\end{solution}

The set of \textbf{irrational numbers} is denoted by $\mathbb{I}$.
The set of \textbf{real numbers} is $\mathbb{R} = \mathbb{Q} \cup \mathbb{I}$.
Note that $\mathbb{Z} \subset \mathbb{Q} \subset \mathbb{R}$.

\subsection*{Ordering of Real Numbers}
If $a,b,c \in \mathbb{R}$ then
\begin{enumerate}
\item $a<b \implies a+c<b+c$

\item $a<b$ and $c>0$ implies $ac<bc$

\item $a<b$ and $c<0$ implies $ac>bc$

\item $a>0$ implies $\frac{1}{a}>0$

\item $0<a<b$ implies $\frac{1}{b}<\frac{1}{a}$
\end{enumerate}

\subsection*{Intervals}

The open interval $(a,b) = \{x \mid a < x < b\}$, closed interval $([a,b])$, half open intervals $(a, b]$, $[a, b)$. It is possible that $a=-\infty$, $b=\infty$.
Draw each interval on the real line.

\begin{example}
  Solve the following inequalities.
  \begin{enumerate}
    \item $\frac{2}{x-1} \ge 5$.
    \begin{solution}
      \textit{It is not right to multiply both sides by $x-1$ and say $5x-5 \le 2$.}
      \[
        \frac{2}{x-1} \ge 5 \iff \frac{2}{x-1} - 5 \ge 0
        \iff \frac{7-5x}{x-1} \ge 0.
      \]
      Now make a sign analysis to get interval $(1, 7/5]$
    \end{solution}

    \item $3x-1 \le 5x+3 \le 2x+15$.
    \begin{solution}
      $-2\le x$ and $x \le 4$.
    \end{solution}
  \end{enumerate}
\end{example}

\subsection*{The absolute value.}
\[
  \abs{x} =
  \begin{cases}
    x, &\text{ if } x\ge 0\\
    -x, &\text{ if } x< 0\\
  \end{cases}
\]

ex. $\abs{3} = \abs{-3} = 3$

Geometrically, $\abs{x}$ is the distance between $x$ and $0$ on the real line. And $\abs{x-y}$ is the distance between $x$ and $y$.

Properties \textit{(can be proved from definition)}:
\begin{enumerate}
  \item $\abs{-x} = \abs{x}$, (Do not fall into the trap $\abs{-x} = x$, this is not always true!)
  \item $\abs{ab} = \abs{a} \abs{b}$,
  \item $\abs{a+b} \le \abs{a} + \abs{b}$, (triangle inequality).
\end{enumerate}
From (2), for any $x$, $x^2 = \abs{x^2} = \abs{x}^2$

If D is a nonnegative number
\[
  \abs{x}=D \implies x=-D \text{ or } x=D,
\]
\[
  \abs{x}<D \implies -D<x<D
\]
\[
  \abs{x}>D \implies x<-D \text{ or } x>D
\]
More generally,
\[
  \abs{x-a}=D \implies x=a-D \text{ or } x=a+D,
\]
\[
  \abs{x-a}<D \implies a-D<x<a+D
\]
\[
  \abs{x-a}>D \implies x<a-D \text{ or } x>a+D
\]

\begin{example}
  Solve $\abs{3x-2} \le 1$.

  Solution.
  \[
    -1 \le 3x-2 \le 1 \implies x \ge 1/3 \text{ and } x \le 1.
  \]
\end{example}

\begin{example}
  Solve the equation $\abs{x+1} > \abs{x-3}$.

  Solution. The distance between $x$ and $-1$ is greater than the distance between $x$ and $3$. So $x>1$.
\end{example}
\end{document}
