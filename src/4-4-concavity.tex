\documentclass[../calc1-main.tex]{subfiles}

\begin{document}

  We say $f$ is \textbf{concave up} on an interval $I$ if $f'$ is increasing on $I$ and \textbf{concave down} on $I$ if $f'$ decreasing on $I$.

  Note that if $f$ is concave up then $f$ lies above its tangents and below its chords while if $f$ is concave down then $f$ lies below its tangents and above its chords.

  If $f$ changes its concavity at $x_0$ then we call $x_0$ and \textbf{inflection point}.

  \begin{theorem}
    Assume $f$ is twice differentiable.
    \begin{enumerate}
      \item[a)] If $f''>0$ on an interval $I$ then $f$ is concave up on $I$,
      \item [b)] If $f''<0$ on an interval $I$ then $f$ is concave down on $I$,
      \item [c)] If $f$ has an inflection point at $x_0$ then $f''(x_0)=0$.
    \end{enumerate}
  \end{theorem}
  Note $f''(x_0) = 0$ does not necessarily mean $x_0$ is an inflection point, for example for $f(x) = x^4$ $f''(0)=0$ while $f$ does not change concavity at $x=0$.

  \begin{minipage}{0.5\textwidth}
    \begin{example}
      Determine the intervals of concavity of $f(x) = x^6 - 10x^4$ and the inflection points of its graph.
    \end{example}
    \begin{solution}
      $f'(x) = 2x^3 (3x^2-20)$, $f''(x) = 30x^2(x-2)(x+2)$. So possible inflection points are $0$, $\pm 2$.

      \begin{table}[H]
        \centering
        \begin{tabular}{c||ccccccc}
          x & & -2 & & 0 & & 2 & \\
          \hline
          $f''$ & + & 0 & - & 0 & - & 0 & + \\
          $f$ & c.up & infl. & c.down & & c.down & infl & c.up
        \end{tabular}
      \end{table}
      The inflection points are $\pm 2$.
    \end{solution}
\end{minipage}%
\begin{minipage}{0.5\textwidth}
  \begin{figure}[H]
    \centering
    \pgfplotsset{soldot/.style={color=black,only marks,mark=*}}
\begin{tikzpicture}
  \begin{axis}[
  axis lines=middle, % left, right, box, center, none
  x=8mm,
  y=0.25mm,
  xmin=-4, xmax=4, ymax= 50,
  title={$f(x)=x^6-10x^4$},
  xlabel=$x$,
  ylabel=$y$,
  xtick={-2, 2},
  ytick={-96},
  grid=major,
  grid style={dashed, gray}
  ]
  \addplot[domain=-3.2:3.2, samples=300,very thick] {x^6-10*x^4};
  \addplot[soldot] coordinates{(-2,-96)(2,-96)(0,0)};
\end{axis}
\end{tikzpicture}
  \end{figure}
\end{minipage}

\begin{minipage}{0.5\textwidth}
  \begin{example}
    Determine the intervals of increase, decrease , the local extreme values and the concavity of $f(x) = x^4-2x^3+1$. Sketch the graph of $f$.
  \end{example}
  \begin{solution}
    $f'(x) = 4x^3 - 6x^2 = 2x^2 (2x-3)$, critical points are $x=0$, $x=3/2$.

    $f''(x) = 12x (x-1)$, possible inflection points are $x=0$, $x=1$.

    \begin{table}[H]
      \centering
      \begin{tabular}{c||ccccccc}
        x & & 0 & & 1 & & 3/2 & \\
        \hline
        $f'$ & - & 0 & - & & - & 0 & + \\
        \hline
        $f''$ & + & 0 & - & 0 & + & & + \\
        \hline
        $f$ & $\searrow$ & & $\searrow$ & & $\searrow$ & min & $\nearrow$ \\
        & c.up & infl & c.down & infl & c.up & & c.up \\
        \hline
      \end{tabular}
    \end{table}
  \end{solution}
\end{minipage}%
\begin{minipage}{0.5\textwidth}
  \begin{figure}[H]
    \centering
    \pgfplotsset{soldot/.style={color=black,only marks,mark=*}}
\begin{tikzpicture}
  \begin{axis}[
  axis lines=middle, % left, right, box, center, none
  x=8mm,
  y=8mm,
  xmin=-2.5, xmax=2.5, ymax=4, ymin=-2,
  title={$f(x)=x^4-2x^3+1$},
  xlabel=$x$,
  ylabel=$y$,
  ticks=none
  ]
  \addplot[domain=-2:2, samples=300, very thick] {x^4-2*x^3+1};
  \addplot[soldot] coordinates{(0,1)(1,0)(3/2,-11/16)};
  \node[above right] at (0,1) {$1$};
  \node[above right] at (1,0) {$1$};
  \node[below] at (3/2,-11/16) {$(\frac{3}{2},-\frac{11}{16})$};
\end{axis}
\end{tikzpicture}
  \end{figure}
\end{minipage}

\subsection*{The Second Derivative Test}
\begin{theorem}
  \begin{enumerate}
    \item[a)] If $f'(x_0) = 0$ and $f''(x_0) < 0$, then $f$ has a local max at $x_0$.
    \item[b)] If $f'(x_0) = 0$ and $f''(x_0) > 0$, then $f$ has a local min at $x_0$.
    \item[c)] If $f'(x_0)= f''(x_0)$, then no conclusion can be drawn.
  \end{enumerate}
\end{theorem}

\begin{minipage}{0.5\textwidth}
  \begin{example}
    Find an classify the critical points of $f(x) = x^2 e^{-x}$.
  \end{example}
  \begin{solution}
    $f'(x) = x(2-x)e^{-x} = 0$, at $x=0$, $x=2$.
    $f''(x) = (2-4x+x^2)e^{-x}$. $f''(0) = 2 >0$ and $f''(2) = -2e^{-2} < 0$.
    Thus $f$ has a local min at $x=0$ and local max at $x=2$.
  \end{solution}
\end{minipage}%
\begin{minipage}{0.5\textwidth}
  \begin{figure}[H]
    \centering
    \pgfplotsset{soldot/.style={color=black,only marks,mark=*}}
\begin{tikzpicture}
  \begin{axis}[
  axis lines=middle, % left, right, box, center, none
  x=12mm,
  y=12mm,
  xmin=-1.5, xmax=4.5,
  title={$f(x)=x^2 e^{-x}$},
  xlabel=$x$,
  ylabel=$y$,
  ticks=none
  ]
  \addplot[domain=-1:3.8, samples=300, very thick] {x^2*exp(-x)};
  \addplot[soldot] coordinates{(0,0)(2,0.54)};
  \node[above ] at (2,0.54) {$(2, 4 e^{-2})$};
\end{axis}
\end{tikzpicture}
  \end{figure}
\end{minipage}

\end{document}
