\documentclass[calc1-main.tex]{subfiles}

\begin{document}

\begin{definition}
  A \textbf{polynomial} is a function $P: \mathbb{R} \to \mathbb{R}$ such that
  \[
    P(x) = a_n x^n + \cdot + a_1 x + a_0.
  \]
  Here $a_n, \dots, a_1$ are called the \textbf{coefficients} of the polynomial. We assume $a_n \neq 0$. The number $n$ is called the \textbf{degree} of the polynomial.
\end{definition}

\begin{example}
  Write polynomials of degree 0, 1 and 2.
\end{example}

Just as the quotient of two integers is called a rational number, the quotient of two polynomials is called a \textbf{rational function}. For example $f(x) = \frac{1-x}{3x^2 + 2x + 1}$ is a rational function.

\begin{theorem}[Euclidean Division]
  Let $p(x)$ (the dividend) of degree $m$ and $d(x)$ (the divisor) of degree $n$ be two polynomials with $m\ge n$.
  Then there are polynomials $q(x)$ (the quotient) of degree $m-n$, and $r(x)$ (the remainder) with $r(x) = 0$ or deg($r(x)$)$=k<n$ such that
  \[
    p(x) = d(x) q(x) + r(x)
  \]
\end{theorem}
\begin{proof}
  The proof is the standard ``long division'' algorithm.
\end{proof}

\begin{example}
  Using the long division algorithm, show that
  \[
    \frac{2x^3-3x^2+3x+4}{x^2+1} = 2x-3 + \frac{x+7}{x^2+1}
  \]
\end{example}

If $p$ is a polynomial and $p(r) = 0$ then $r$ is called a \textbf{root} of $p$.

\begin{theorem}[The Fundamental Theorem of Algebra]
   Every non-constant polynomial must have a root which may be a complex number.
\end{theorem}

\begin{example}
  $x^2+1$ has no real roots. Its roots are $i=\sqrt{-1}$ and $-i$.
\end{example}

\begin{theorem}
  If r is a root of the polynomial $p$ then
  \[
    p(x) = (x-r) q(x),
  \]
  for some polynomial $q$ whose degree is 1 less than $p$.
\end{theorem}

\begin{theorem}
  Every polynomial of degree $n$ has exactly $n$ (not necessarily distinct) roots.
  \[
    p(x) = (x-x_1)^{m_1}(x-x_2)^{m_2}\cdots(x-x_k)^{m_k}
  \]
  with $m_1 + m_2 + \cdots + m_k = \text{deg}(p(x))$.
\end{theorem}
\begin{proof}
  Proof follows from the Fundamental Theorem of Algebra and the above theorem.
\end{proof}
\subsection*{Roots of Quadratic Polynomials}
To obtain the solutions of
\[
  A x^2 + B x + C = 0, \qquad A \neq 0
\]
Divide by $A$ and complete to square
\[
  \left(x + \frac{B}{2A}\right)^2 = \frac{B^2}{4A^2} - \frac{C}{A} = \frac{B^2-4AC}{4A^2},
\]
Taking the square root of both sides gives the quadratic formula
\[
  x = \frac{-B \pm \sqrt{B^2 - 4AC}}{2A}.
\]

\textit{A form of this formula is known since B.C. 2000 by Babylonians.}

The quantity $D=B^2 - 4AC$ is called the \textbf{discriminant} of the quadratic equation.

\begin{itemize}
  \item If $D>0$ then there are two distinct real roots,
  \item If $D=0$ then there is 1 root of multiplicity 2,
  \item If $D<0$ then there are two complex conjugate roots.
\end{itemize}

\begin{example}
  Find the roots of the polynomials: (a) $x^2+x-1$, (b) $9x^2 -6x+1$, (c) $2x^2+x+1$.
\end{example}

\subsection*{Misc Factorings}

\begin{itemize}
  \item Difference of squares:
  \[
    x^2 - a^2 = (x-a)(x+a)
  \]
  \item Difference of cubes:
  \[
    x^3 - a^3 = (x-a)(x^2+ax+a^2)
  \]
  \item Difference of nth powers
  \[
    x^n - a^n = (x-a)(x^{n-1}+ a x^{n-2} + a^2 x^{n-3} + \cdots + a^{n-2}x + a^{n-1})
  \]
  \item If $n$ is an odd integer then $x+a$ is a factor of $x^n+a^n$,
  \[
    x^n + a^n = (x+a)(x^{n-1} - a x^{n-2} + a^2 x^{n-3} - \cdots - a^{n-2}x + a^{n-1})
  \]
\end{itemize}
\end{document}
