\documentclass[../calc1-main.tex]{subfiles}

\begin{document}
	Let $f(x) = \sqrt{4 - x^2}$. Domain of $f$ is $[-2, 2]$.
	\begin{itemize}
		\item $x = -2$ is the left end point of Dom($f$).
		\item $x = 2$ is the right end point of Dom($f$).
		\item Any $x$ with $-2 < x < 2$ is called an interior point of Dom($f$).
	\end{itemize}

	\begin{definition}
		A function $f$ is \textbf{continuous} at an interior point $c$ of its domain if
		\[
			\lim_{x \to c} f(x) = f(c)
		\]
		$f$ is continuous at its left endpoint $c$ if
		\[
			\lim_{x \to c+} f(x) = f(c)
		\]
		$f$ is continuous at its right endpoint $c$ if
		\[
			\lim_{x \to c-} f(x) = f(c)
		\]
	\end{definition}

	The following theorem gives an alternative definition of continuity which is sometimes useful.
	\begin{theorem}\label{alternative continuity}
	A function $f$ is \textbf{continuous} at an interior point $c$ of its domain if and only if
	\[
		\lim_{h \to 0} f(c+h) = f(c)
	\]
	$f$ is continuous at its left endpoint $c$ if
	\[
		\lim_{h \to 0+} f(c+h) = f(c)
	\]
	$f$ is continuous at its right endpoint $c$ if
	\[
		\lim_{h \to 0-} f(c+h) = f(c)
	\]
\end{theorem}
\begin{proof}
	Let $h=x-c$. Then $x \to c$ if and only if $h \to 0$. So $\lim_{h \to 0} f(c+h) = f(c)$ is the same as $\lim_{h \to 0} f(c+h) = f(c)$.
\end{proof}
Note that $f$ is discontinuous at $c$ if
\begin{itemize}
	\item[i)] either $\lim_{x \to c} f(x)$ does not exist.
	\item[ii)] or $\lim_{x \to c} f(x)$ exists but is not equal to $f(c)$.
\end{itemize}

\begin{figure}[H]
	\centering
	\pgfplotsset{soldot/.style={color=blue,only marks,mark=*}}
\pgfplotsset{holdot/.style={color=blue,fill=white,only marks,mark=*}}

\begin{tikzpicture}
\begin{axis}[
xtick={3,5,7},
yticklabels={,,},
xticklabels={$a$, $b$, $c$}
]
\addplot[domain=0:3,blue, thick] {x*x};
\addplot[domain=3:5,blue, thick] {x};
\addplot[domain=5:9,blue, thick] {5};
\draw[dotted] (axis cs:3,9) -- (axis cs:3,0);
\draw[dotted] (axis cs:5,5) -- (axis cs:5,0);
\draw[dotted] (axis cs:7,5) -- (axis cs:7,0);
\addplot[holdot] coordinates{(0,0)(3,3)(5,5)};
\addplot[soldot] coordinates{(3,9)(5,1)(7,5)};
\end{axis}
\end{tikzpicture}

	\caption{f is discontinuous at $a$ because of (ii) and discontinuous at $b$ because of (i). f is continuous at $c$.}
\end{figure}

\begin{definition}
	$f$ is called a continuous function if $f$ is continuous at every pt of its domain.
\end{definition}

\begin{example}
	$f(x) = \sqrt{4 - x^2}$ is continuous at every point of its domain. So it is a continuous function.
	\begin{figure}[H]
		\centering
		\begin{tikzpicture}[scale=0.8]
\begin{axis}[
  title={$f(x)=\sqrt{4-x^2}$},
  xtick={-2,2},
  ytick={2},
  y tick label style={anchor=south east},
  axis lines =middle,
  xmin=-4, xmax=4,
  ymin=-1, ymax=3,
  x=8mm,
  y=8mm
]
  \addplot[domain=-2:2,samples=300,thick] {sqrt(4-x^2)};
\end{axis}
\end{tikzpicture}

	\end{figure}
\end{example}


According to this definition $f(x) = \frac{1}{x}$ is also continuous!!! $0$ is not in domain of $f$. So we say $f$ is undefined rather than discontinuous at $0$.

\textbf{There are lots of continuous functions:}
\begin{itemize}
	\item polynomials,
	\item rational functions,
	\item rational powers $x^{m/n}$
	\item trigonometric functions
	\item absolute value function $\abs{x}$
\end{itemize}

\begin{theorem}
	If $f$ and $g$ are continuous at $c$ then
	\begin{itemize}
		\item $f + g$, $f - g$, $f g$, are continuous at $c$,
		\item if $k$ is constant then $k f$ is continuous at $c$,
		\item $\dfrac{f}{g}$ continuous at $c$ provided that $g(c) \neq 0$.
		\item $f(x)^{1/n}$ continuous at c provided that $f(c)>0$ if $n$ is even.
	\end{itemize}
\end{theorem}

\begin{proof}
	Let's prove that if $f$ and $g$ are continuous at $c$ then so is $f+g$. If $f$ and $g$ are continuous at $c$ then
	\[
		\lim_{x \to c} f(x) = f(c), \qquad
		\lim_{x \to c} g(x) = g(c), \qquad
	\]
	By the limit rule,
	\[
		\lim_{x \to c} (f(x) + g(x)) =
		\lim_{x \to c} f(x) + \lim_{x \to c} g(x) =
		f(c) + g(c).
	\]
	The other proofs are similar.
\end{proof}
\textbf{Composites of continuous functions are continuous}

If $g$ is continuous at $c$ and $f$ is continuous at $g(c)$ then $f \circ g$ is continuous at $c$. In other words,
\[
	\lim_{x \to c} f(g(x)) = f(\lim_{x \to c} g(x)) = f(g(c)).
\]

\begin{example}
	Find $m$ so that
	\[
		g(x) = \begin{cases}
		x-m, &\text{ if }x < 3,\\
		1-mx, &\text{ if }x \geq 3\\
	\end{cases}
\]
is continuous for all $x$.
\end{example}

% \textbf{Continuous extensions and removable discontinuities}

% If $f$ is not defined at $c$ but $\lim_{x \to c} f(x) = L$ is defined then we can define a new function
% \[
%     F(x) =
%     \begin{cases}
%         f(x), &\text{ if } x \neq c\\
%         L, &\text{ if } x =c\\
%     \end{cases}
% \]
% $F$ is continuous at $c$ and is called \textbf{continuous extension} of $f$ to $x=c$.
% \begin{example}
%     The function $f(x) = \dfrac{x^2 - x}{x^2 - 1}$ is not defined at $x=1$ but has a continuous extension $F(x) = \dfrac{x}{x+1}$ to $x=1$.
% \end{example}

% If a function is undefined or discontinuous at $c$ but can be redefined at $c$ then we say $f$ has a \textbf{removable discontinuity} at $c$.

% \begin{example}
%     The function $f(x) = \dfrac{1}{x^2}$ is not defined at 0 but there is no way of redefining $f$ at 0 so that $f$ becomes continuous at 0.
% \end{example}
\subsection*{Continuity of Trigonometric Functions}
\begin{theorem}
	$\sin x$ and $\cos x$ are continuous at $x=0$, i.e.
	\[
		\lim_{x \to 0} \sin x = \sin 0 = 0, \qquad
		\lim_{x \to 0} \cos x = \cos 0 = 1.
	\]
\end{theorem}
\begin{proof}
	~\newline
	\begin{minipage}{0.3\textwidth}
		\begin{align*}
			& \abs{1-\cos \theta}=\abs{AQ} \le \abs{AP} \le \theta, \\
			& \abs{\sin \theta} = \abs{PQ} \le \abs{AP} \le \theta
		\end{align*}
		In other words, $-\theta \le \sin \theta \le \theta$ and using the squeeze theorem we get $\lim_{\theta \to 0}  \sin \theta=0$. Similarly, we get $\lim_{\theta \to 0} 1-\cos \theta = 0$ or $\lim_{\theta \to 0} \cos \theta = 1$.
	\end{minipage}
	\begin{minipage}{0.5\textwidth}
		\begin{figure}[H]
			\centering
			\pgfplotsset{soldot/.style={color=blue,only marks,mark=*}}
\pgfplotsset{holdot/.style={color=blue,fill=white,only marks,mark=*}}

\begin{tikzpicture}
  \begin{axis}[
  ymin=-1.5,
  ymax=1.5,
  xmin=-1.5,
  xmax=2,
  ticks=none,
  x=20mm,
  y=20mm,
  axis lines = middle,
  disabledatascaling
  ]
  \draw[black, thick] (axis cs:0,0) circle [radius=1];
  \draw[black] (axis cs:0.6,0) -- (axis cs: 0.6,0.8);
  \draw[black] (axis cs:0,0) -- (axis cs: 0.6,0.8);
  \draw[black] (axis cs:1,0) -- (axis cs: 0.6,0.8);
  \addplot[soldot] coordinates{(0,0)(0.6,0)(0.6,0.8)(1,0)};
  \node[above right] at (axis cs: 1,0) {$A$};
  \node[above] at (axis cs: 1.2,0.8) {$P=(\cos \theta, \sin \theta)$};
  \node[below] at (axis cs: 0.6,0) {$Q$};
  \node[right] at (axis cs: 0.8,0.6) {$\theta$};
  \draw [->] (axis cs:.2,0) arc [radius=.2,start angle=0,end angle=53.1];
  \node[right] at (axis cs: 0.15,0.15) {\tiny$\theta$};
  \node[above] at (axis cs: 0.25, 0.35) {\tiny 1};
\end{axis}
\end{tikzpicture}

		\end{figure}
	\end{minipage}
\end{proof}

\begin{theorem}
	$\sin x$ and $\cos x$ are continuous for all $x$.
\end{theorem}
\begin{proof}
	By Theorem~\ref{alternative continuity}, we need to prove $\lim_{h \to 0} \sin (x+h) = \sin x$ for any $x$.
	\[
		\lim_{h \to 0} \sin(x+h) =
		\lim_{h \to 0} \sin x \cos h + \cos x \sin h =
		\sin x \lim_{h \to 0} \cos h + \cos x \lim_{h \to 0} \sin h = \sin x.
	\]
	Prove the continuity of $\cos x$ as an exercise.
\end{proof}
\subsection*{Extreme Value Theorem}

\begin{theorem}
	If $f$ is continuous on the closed interval $[a, b]$ then there exist numbers $p$ and $q$ in the interval $[a,b]$ s.t.
	\[
		f(p) \leq f(x) \leq f(q)
	\]
	for all $x$ in $[a, b]$.
	$f(p)$ is called the \textbf{absolute minimum value} and $f(q)$ is called the \textbf{absolute maximum value}.
\end{theorem}

Extreme value theorem is an existence theorem. It only guarantees the existence of $p$ and $q$ but does not tell how to actually find them.

We say a function $f$ is \textbf{bounded} if there exists $M$ and $N$ such that $M \le f(x) \le N$ for all $x$ in the domain of $f$. Extreme value theorem says that continuous functions on closed intervals must be bounded.

\begin{example}
	The conclusions of the theorem may fail if the function $f$ is not continuous or the interval is not closed.

	\begin{figure}[H]
		\centering
		\begin{subfigure}[t]{0.4\textwidth}
			\pgfplotsset{soldot/.style={color=blue,only marks,mark=*}}
\pgfplotsset{holdot/.style={color=blue,fill=white,only marks,mark=*}}

\begin{tikzpicture}
\begin{axis}[
  x=12mm,
  y=8mm,
  ymax=3,
  xmax= 1.5,
  xtick={1},
  ytick={0},
  axis lines = left,
]
\addplot[domain=0.01:1,blue, thick] {1/x};
\draw[dotted] (axis cs:1,1) -- (axis cs:1,0);
\addplot[holdot] coordinates{(0,0)(1,1)};
\end{axis}
\end{tikzpicture}

			\caption{The function $f(x) = 1/x$ on the open interval $(0,1)$ is continuous but unbounded and has no minimum and no maximum.}
		\end{subfigure}
		\quad
		\begin{subfigure}[t]{0.4\textwidth}
			\input{figures/1-3-ctsFuncsOnClosedInts1.tikz}
			\caption{The function $f(x) = x$ on $(0,1)$ is discontinuous, bounded and has no minimum and no maximum.}
		\end{subfigure}
	\end{figure}

	\begin{figure}[H]
		\centering
		\begin{subfigure}[t]{0.4\textwidth}
			\pgfplotsset{soldot/.style={color=blue,only marks,mark=*}}
\pgfplotsset{holdot/.style={color=blue,fill=white,only marks,mark=*}}

\begin{tikzpicture}
\begin{axis}[
  x=22mm,
  y=22mm,
  xmax=1.2,
  xtick={1},
  ytick={0},
  axis lines = left,
]
\addplot[domain=0:1, thick] {x};
\draw[dotted] (axis cs:1,1) -- (axis cs:1,0);
\addplot[holdot] coordinates{(1,1)};
\addplot[soldot] coordinates{(0,0)(1,0)};
\end{axis}
\end{tikzpicture}

			\caption{This function is defined on the closed interval $[0,1]$, discontinuous, has a minimum but no maximum.}
		\end{subfigure}
		\quad
		\begin{subfigure}[t]{0.4\textwidth}
			\pgfplotsset{soldot/.style={color=blue,only marks,mark=*}}
\pgfplotsset{holdot/.style={color=blue,fill=white,only marks,mark=*}}

\begin{tikzpicture}
\begin{axis}[
  x=22mm,
  y=22mm,
  xmax=1.2,
  xtick={1},
  ytick={0},
  axis lines = left,
]
\addplot[domain=0:0.5, thick] {x+0.5};
\addplot[domain=0.5:1, thick] {x-0.5};
\draw[dotted] (axis cs:0.5,1) -- (axis cs:0.5,0);
\addplot[holdot] coordinates{(0.5,1)(0.5,0)};
\addplot[soldot] coordinates{(0.5,0.5)(1,0.5)};
\end{axis}
\end{tikzpicture}

			\caption{This function is defined on the closed interval $[0,1]$, discontinuous, bounded, has no minimum and no maximum.}
		\end{subfigure}
	\end{figure}
\end{example}

\subsection*{Intermediate Value Theorem}
\begin{theorem}[Intermediate Value Theorem]
	If $f$ is continuous on $[a, b]$ and if $s$ is between $f(a)$ and $f(b)$ then there exists $c$ in $[a, b]$ s.t. $f(c) = s$.
\end{theorem}

\begin{figure}[H]
	\centering
	\pgfplotsset{soldot/.style={color=blue,only marks,mark=*}}
\pgfplotsset{holdot/.style={color=blue,fill=white,only marks,mark=*}}

\begin{tikzpicture}
\begin{axis}[
  x=24mm,
  y=36mm,
  xmin=-.2,
  ymin=0.2,
  xmax=1.8,
  xtick={0,1.08,1.3},
  xticklabels={$a$,$c$,$b$},
  ytick={0.6},
  yticklabels={$s$},
  axis lines = left,
]
\addplot[domain=0:1.3,blue,thick] {x^3-x^2+0.5};
\draw[dotted, thick] (axis cs:-.2,.6) -- (axis cs:1.08,.6);
\draw[dotted, thick] (axis cs:1.08,.6) -- (axis cs:1.08,0.2);
\draw[dotted, thick] (axis cs:0,0.2) -- (axis cs:0,0.5);
\draw[dotted, thick] (axis cs:1.3,0.2) -- (axis cs:1.3,1);
\addplot[soldot] coordinates{(0,0.5)(1.3,1)};
\end{axis}
\end{tikzpicture}

	\caption{Illustration of the intermediate value theorem.}
\end{figure}

\begin{example}
	If a child grows from 1 m to 1.5 m between the ages of two and six years, then, at some time between two and six years of age, the child's height must have been 1.23 m.
\end{example}
In particular, a continuous function on a closed interval takes every value between its minimum $m$ and maximum $M$. Hence its range is a closed interval $[m, M]$.
\begin{example}
	Show that the equation $x^3 - x - 1 = 0$ has a solution in the interval $[1, 2]$.
\end{example}

\begin{solution}
	$f(x) = x^3 - x - 1$ is a polynomial and hence continuous. $f(1) = -1$ and $f(2) = 5$. Since $0$ lies between $-1$ and $5$, the intermediate value theorem assures us that there must be a number $c$ in $[1, 2]$ such that $f(c) = 0$.
\end{solution}

\subsection*{Bisection Algorithm}
Intermediate Value Theorem is also an existence theorem. It does not say how to find $c$ in its statement. Let's try to better estimate the root of previous example. Write $f(x) = x^3 - x - 1$ and try to find a smaller interval where a root lies of
\[
	f(x) = 0.
\]
We know that a root lies in $[1, 2]$, if say that the root is $1.5$ the maximum error will be 0.5.

Now $f(1.5)=0.875 > 0$. So a root lies in $[1, 1.5]$, and if we say the root is $1.25$ then the maximum error will be $0.25$.

If this is not sufficient then compute $f(1.25)=-0.2969$, now if we say the root is $1.375$ then the error is less than $0.125$.

Next step is $f(1.1375) = 0.2246$. So a root must lie in $[1.25, 1375]$. The error is less than $0.0625$ if we say the root is $1.315$.

Going this way, we find the approximations, $1.3438$, $1.3282$, $1.3204$. Hence the root must lie in $[1.3204, 1.3282]$. So the first two decimal digits of the root are $1.32$.
% 1 is the whole number digit.

In engineering, you almost never get exact results. All you can do is lower your error below an acceptable threshold.

\subsection*{Optional Issues}
\textbf{Is there a function which is continuous only at a single point? Yes!}
\begin{example}
  \[
    f(x) =
    \begin{cases}
      x, & \text{if $x$ is a rational number} \\
      0, & \text{otherwise}
    \end{cases}
  \]
  is continuous only at $x=0$.
\end{example}
This also answers the following question

\textbf{If a function is continuous at point, is it continuous in some open interval around that point? NO!}

\end{document}
