\documentclass[12pt]{memoir}
\setcounter{chapter}{-1}
\chapterstyle{bianchi}

\usepackage{subfiles} % to use subfiles
\usepackage[top=2.5cm, left=1.5cm, right=1.5cm, bottom=2cm]{geometry}
\usepackage{fourier}
\usepackage{multicol}
\usepackage{sidecap}
\usepackage[usenames,dvipsnames,svgnames]{xcolor}
\usepackage{pgfplots}
\usepackage{multicol}
\pgfplotsset{compat=1.12}

\usepackage{color}
\usepackage{amsmath}
\usepackage{amsthm}
\usepackage{amsfonts}
\usepackage{subcaption}
\usepackage{xfrac}
\usepackage{hyperref}
\usepackage{bigints} % for big integrals
\usepackage[utf8]{inputenc}
\usepackage{hyperref}
\usepackage{float} % to force figure placement in text. use the [H] option in the figure.
\usepackage{graphicx}
\graphicspath{{./figures/}}

\usepackage{tikz}
\usetikzlibrary{shadows}
\usetikzlibrary{shapes}
\usetikzlibrary{decorations}
\usetikzlibrary{patterns}
\usetikzlibrary{shapes.geometric,fit} % used in P-4-functions.tex
\usepgfplotslibrary{fillbetween} %  5-4-ex2.tikz
\tikzset{>=stealth} %% makes all arrows the same
\colorlet{penColor}{green!50!black} % Color of a curve in a plot
\colorlet{background}{Apricot!7} % Color of the page
\colorlet{gridColor}{gray!30} % Color of grid in a plot
\hypersetup{
    colorlinks,
    citecolor=blue,
    filecolor=blue,
    linkcolor=blue,
    urlcolor=blue
}


% Colors
\newcommand{\red}[1]{\textcolor{red}{#1}}
\newcommand{\blue}[1]{\textcolor{blue}{#1}}

% Operators
\newcommand{\abs}[1]{\left\lvert #1 \right\rvert}
\newcommand{\norm}[1]{\left \lVert #1 \right \rVert}
\newcommand{\inpr}[2]{\langle #1, #2 \rangle}
\renewcommand\Re{\operatorname{Re}}
\renewcommand\Im{\operatorname{Im}}

% ENVIRONMENTS
\newtheorem{theorem}{Theorem}
\newtheorem{corollary}{Corollary}
\newtheorem{definition}{Definition}
\newtheorem*{example}{Example}
\newtheorem*{exercise}{Exercises}
\newtheorem{lemma}{Lemma}
\newtheorem{notation}{Notation}
\newtheorem{proposition}{Proposition}
\newtheorem*{remark}{Remark}
\newtheorem*{solution}{Solution}

\title{Calculus I Lecture Notes}
\author{Taylan Şengül}


\begin{document}
\maketitle
\tableofcontents
\chapter{Precalculus}
    \section{Sets}
    \subfile{P-0-sets.tex}
    \section{Real Numbers}
    \subfile{P-1-realnumbers.tex}
    \section{Cartesian Coordinates}
    \subfile{P-2-cartesian.tex}
    \section{Quadratic Equations}
    \subfile{P-3-quadratics.tex}
    \section{Functions and Their Graphs}
    \subfile{P-4-functions.tex}
    \section{Operations on Functions}
    \subfile{P-5-operfunctions.tex}
    \section{Polynomials and Rational Functions}
    \subfile{P-6-polys.tex}


\chapter{Limits and Continuity}
    \section{Informal definition of limits}
    \subfile{1-1-limit.tex}
    \section{Limits at Infinity and Infinite Limits}
    \subfile{1-2-infiniteLimits.tex}
    \section{Continuity}
    \subfile{1-3-continuity.tex}
    \section{Formal definition of Limit}
    \subfile{1-4-formal.tex}
    \section{Review Problems}
    \subfile{1-5-limit-review.tex}
%     \section{One Sided Limits and Limits at Infinity}
%     \subfile{1-6-limit-variations}
%     \section{Properties of limits}
%     \subfile{1-2-properties-limits.tex}


\chapter{Differentiation}
    \section{Tangent Lines and Their Slopes}
    \subfile{2-1-tangent}
    \section{Derivative}
    \subfile{2-2-derivative}
    \section{Differentiation Rules}
    \subfile{2-3-diffrules}
    \section{Chain Rule}
    \subfile{2-4-chain}
    \section{Derivatives of Trigonometric Functions}
    \subfile{2-5-trig}
    \section{Higher Order Derivatives}
    \subfile{2-6-higher}
    \section{Mean Value Theorem}
    \subfile{2-8-meanValueTheorem}
    \section{Implicit Differentiation}
    \subfile{2-9-implicit}
    \section{Exam 1 Review}
    \subfile{exam1review}
    % \section{Antiderivatives}
    % \subfile{2-10-antider}

\chapter{Transcendental Functions}
    \section{Inverse Functions}
    \subfile{3-1-inverse}
    \section{Exponential and Logarithmic Functions}
    \subfile{3-2-exp}
    \section{The Natural Logarithm and Exponential}
    \subfile{3-3-naturallog}
    \section{The Inverse Trigonometric Functions}
    \subfile{3-4-inversetrig}

\chapter{Applications of Derivatives}
    \section{Related Rates}
    \subfile{4-1-related.tex}
    \section{Indeterminate Forms}
    \subfile{4-2-indeterminate-forms.tex}
    \section{Extreme Values}
    \subfile{4-3-extreme.tex}
    \section{Concavity and Inflections}
    \subfile{4-4-concavity.tex}
    \section{Graphs of Functions}
    \subfile{4-5-graph.tex}
    \section{Extreme Value Problems}
    \subfile{4-6-extreme-problems.tex}
    \section{Linear Approximation}
    \subfile{4-7-linear-approximation.tex}
    \section{Exam 2 Review}
    \subfile{exam2review}


\chapter{Integration}
    \section{The Definite Integral}
    \subfile{5-1-definite-integral.tex}
    \section{The Fundamental Theorem of Calculus}
    \subfile{5-2-FTC.tex}
    \section{The Method of Substitution}
    \subfile{5-3-substitution.tex}
    \section{Areas of Plane Regions}
    \subfile{5-4-areas.tex}
    \section{Integration by Parts}
    \subfile{5-5.intByParts.tex}
    \section{Integrals of Rational Function}
    \subfile{5-6-integralOfRationalFuncs.tex}
    \section{Inverse Substitutions}
    \subfile{5-7-inverseSubstitutions.tex}
    \section{Improper Integrals}
    \subfile{5-8-improper.tex}
\end{document}
