\documentclass[../main.tex]{subfiles}

\begin{document}

A function has an \textbf{absolute maximum value} $f(x_0)$ if $f(x) < f(x_0)$ holds for every $x$ in its domain.

Similarly, define \textbf{absolute minimum value}.

If it has an absolute min/max, then that value may be achieved at more than one point. For example the function $\cos x$ attains its absolute max at $x = 2 n \pi $ for any integer $n$.

A function may or may not have an absolute min/max value. For example the function $f(x) = x$, $0<x<1$ does not have an absolute maximum or minimum.

Recall from the section on continuous functions that,

\texttt{A continuous function defined on a closed and bounded interval must have an absolute maximum and an absolute minimum.}

Maximum and minimum values of a function are collectively referred to as \textbf{extreme values}.

Function $f$ has a \textbf{local maximum} value $f(x_0)$ if there exists $h > 0$ such that $f(x) \le f(x_0)$ whenever $x$ is in the domain of $f$ and $\abs{x-x_0}<h$.

Similarly we define \textbf{local minimum}.

We define \textbf{critical points} of $f$ where $f'(x)=0$, \textbf{singular points} of $f$ where $x$ is in domain of $f$ and $f'(x)$ does not exist.

Following theorem says where the extreme values are located.
\begin{theorem}
  If the function $f$ is defined on an interval $I$ and has a local max or local min at $x=x_0$ then $x_0$ \textbf{must} be either a critical point, a singular point or an endpoint of the interval.
\end{theorem}

\begin{proof}
  If $f(x_0)$ is a local extrema and $x_0$ is not an endpoint or singular point, then $f'(x_0) =0$. Otherwise, either $f'(x_0)>0$ which means $f$ is increasing at $x_0$ or $f'(x_0)<0$ which means $f$ is decreasing at $x_0$ so that $f(x_0)$ is neither a local min nor local max.
\end{proof}

This theorem does not say $f$ must have a local min/max at at every singular, critical or endpoint. For example for $f(x) = x^3$, $f'(0) = 0$ but $f(0)$ is not an extremum value.

\begin{example}
  Find the maximum and minimum values of the function
  $g(x) = x^3 - 3x^2 - 9x +2$ on the interval $-2 \le x \le 2$.
\end{example}
\begin{minipage}{0.5\textwidth}
  \begin{solution}
  $g$ is a continuous function defined on a closed and bounded interval so it must have an absolute minimum and absolute maximum.

  Since $g$ is a polynomial, it can't have singular points.

  $g'(x) = 3(x^2-2x-3)=3(x+1)(x-3)$. $g'(x)=0$ if $x=-1$ or $x=3$. $x=3$ is not in the domain, so we ignore it.

  We check the values of $g(x)$ at endpoints and critical points, $g(-2) = 0$, $g(-1)=7$, $g(2)=-20$. The maximum value is 7, the minimum value is -20.
\end{solution}
\end{minipage}%
\begin{minipage}{0.5\textwidth}
  \begin{figure}[H]
    \centering
    \pgfplotsset{soldot/.style={color=black,only marks,mark=*}}
\begin{tikzpicture}
  \begin{axis}[
  axis lines=middle, % left, right, box, center, none
  x=10mm,
  y=1.5mm,
  xmin=-2.5, ymax= 12, ymin=-24,
  title={$f(x)=x^3-3x^2-9x+2$},
  xlabel=$x$,
  ylabel=$y$,
  ticks=none
  ]
  \addplot[domain=-2:2, very thick] {x^3-3*x^2-9*x+2};
  \addplot[soldot] coordinates{(-2,0)(-1,7)(2,-20)};
  \node[below] at (-2,0) {$-2$};
  \node[above] at (-1,7) {$(-1, 7)$};
  \node[left] at (2,-20) {$(2, -20)$};
\end{axis}
\end{tikzpicture}
  \end{figure}
\end{minipage}

\begin{example}
  Find the maximum and minimum values of $h(x)=3x^{2/3}-2x$ on the inteval $[-1,1]$.
\end{example}
\begin{minipage}{0.5\textwidth}
  \begin{solution}
  $h'(x) = 2(x^{-1/3} - 1)$. $h'(0)$ is undefined, $0$ is a singular point of $h$. $h$ has a critical point at $x=1$ which is also an endpoint.

  $h(-1) = 5$, $h(0)=0$, $h(1) = 1$.
  $h$ has maximum value $5$ and minimum value $0$.
\end{solution}
\end{minipage}%
\begin{minipage}{0.5\textwidth}
  \begin{figure}[H]
    \centering
    \pgfplotsset{soldot/.style={color=black,only marks,mark=*}}
\begin{tikzpicture}
  \begin{axis}[
  axis lines=middle, % left, right, box, center, none
  x=30mm,
  y=6mm,
  xmin= -1.2, ymax = 6,
  title={$h(x)=x^{2/3}-2x$},
  xlabel=$x$,
  ylabel=$y$,
  ticks=none
  ]
  \addplot[domain=-1:1, samples=300, very thick] {3*abs(x)^(2/3)-2*x};
  \addplot[soldot] coordinates{(-1,5)(0,0)(1,1)};
  \node[right] at (-.7,5) {$(-1, 5)$};
  \node[below] at (0,0) {$0$};
  \node[above left] at (1,1) {$(1, 1)$};
\end{axis}
\end{tikzpicture}
  \end{figure}
\end{minipage}

\subsection*{The first derivative test}
By investigating the sign of the first derivative we can determine whether an extrema is a local minimum or local maximum.

\begin{example}
  Find the local and absolute extreme values of $f(x) = x^4-2x^2-3$ on the interval $[-2, 2]$. Sketch the graph of $f$.
\end{example}
\begin{minipage}{0.5\textwidth}
  \begin{solution}
  $f'(x) = 2x(x^2-1) = 4x(x-1)(x+1)$. The critical points are $0, -1, 1$. There are no singular points.

  \begin{table}[H]
    \centering
    \begin{tabular}{c||ccccccccc}
      x & -2 & & -1 & & 0 & & 1 & & 2 \\
      \hline
      $f'$ & & - & & + & & - & & + & \\
      $f$ & max & $\searrow$ & min & $\nearrow$ & max & $\searrow$ & min& $\nearrow$ & max
    \end{tabular}
  \end{table}

  $f(-2) = f(2) = 5$, $f(-1) = f(1) = -4$, $f(0) = -3$.

  Since $f$ is continuous and defined on a closed and bounded interval, it must have an absolute min/max. So $5$ is the absolute maximum and $-4$ is the absolute minimum.
\end{solution}
\end{minipage}%
\begin{minipage}{0.5\textwidth}
  \begin{figure}[H]
    \centering
    \pgfplotsset{soldot/.style={color=black,only marks,mark=*}}

\begin{tikzpicture}
  \begin{axis}[
  axis lines=middle, % left, right, box, center, none
  ticks=none,
  x=12mm,
  y=5mm,
  ymin=-5, xmax=2.5,
  title={$f(x)=x^4-2x^2-3$},
  xlabel=$x$,
  ylabel=$y$,
  ]
  \addplot[domain=-2:2, samples=300, very thick] {x^4-2*x^2-3};
  \addplot[soldot] coordinates{(-1,-4)(1,-4)(-2,5)(2,5)(0,-3)};
  \node[below] at (-1,-4) {$(-1, -4)$};
  \node[below] at (1,-4) {$(1, -4)$};
  \node[above] at (-2,5) {$(-2, 5)$};
  \node[above] at (2,5) {$(2, 5)$};
  \node[above right] at (0, -3) {$-3$};
\end{axis}
\end{tikzpicture}
  \end{figure}
\end{minipage}

\begin{example}
  Locate all extreme values of $f(x) = x\sqrt{2-x^2}$. Determine whether any of these extreme values are absolute. Sketch the graph.
\end{example}
\begin{minipage}{0.5\textwidth}
  \begin{solution}
    Note that $f$ has domain $[-\sqrt{2}, \sqrt{2}]$.
    $f'(x) = -2\frac{x^2-1}{\sqrt{2-x^2}}$. Critical points are $\pm1$. Singular points are $\pm \sqrt{2}$ and endpoints are also $\pm \sqrt{2}$.

    $f(\pm \sqrt{2}) = 0$, $f(-1) = -1$, $f(1) = 1$. Since $f$ is continuous on a closed bounded interval it must have maximum value $1$ and minimum value $-1$.
    \begin{table}[H]
      \centering
      \begin{tabular}{c||ccccccc}
        x & $-\sqrt{2}$ & & -1 & &  1 & & $\sqrt{2}$ \\
        \hline
        $f'$ & & - & & + & & - & \\
        $f$ & max & $\searrow$ & min & $\nearrow$ & max & $\searrow$ & min
      \end{tabular}
    \end{table}

  \end{solution}
\end{minipage}%
\begin{minipage}{0.5\textwidth}
  \begin{figure}[H]
      \centering
      \pgfplotsset{soldot/.style={color=black,only marks,mark=*}}
\begin{tikzpicture}
  \begin{axis}[
  axis lines=middle, % left, right, box, center, none
  x=12mm,
  y=12mm,
  ymin=-1.8, ymax=1.8, xmin=-1.5, xmax=2,
  title={$f(x)=x\sqrt{2-x^2}$},
  xlabel=$x$,
  ylabel=$y$,
  ticks=none
  ]
  \addplot[domain=-1.414:1.414, samples=100, very thick] {x*sqrt(2-x^2)};
  \addplot[soldot] coordinates{(-1.41,0)(1.41,0)(-1,-1)(1,1)};
  \node[above] at (-1.41,0) {$-\sqrt{2}$};
  \node[below left] at (1.41,0) {$\sqrt{2}$};
  \node[below] at (-1,-1) {$(-1, 1)$};
  \node[above] at (1,1) {$(1, 1)$};
\end{axis}
\end{tikzpicture}
    \end{figure}
\end{minipage}

% \begin{figure}[H]
%   \centering
%   \includegraphics[width=0.45\textwidth]{4-3-local-extrema.png}
% \end{figure}

\end{document}