\documentclass[../calc1-main.tex]{subfiles}

\begin{document}

  We learned to find the slope of a curve that is the graph of a function. But not all curves are graphs of functions, for example the circle $x^2 + y^2 = 1$.

  Curves are graphs of equations in two variables
  \[
    F(x, y) = 0.
  \]

  For the circle $F(x, y) = x^2 + y^2 -1$.


  \begin{example}
    Find the slope of the circle $x^2 + y^2 = 25$ at the point $(3, -4)$.
  \end{example}
  \begin{solution}
    1st method. Solve the equation $x^2+y^2=1$ for $y$. There are two solutions $y_{1, 2} = \pm \sqrt{25 - x^2}$. The point lies on the graph of $y_2$. Take derivative of $y_2$.

    2nd method. To differentiate with respect to $x$ treat $y$ as a function of $x$ and use Chain Rule.
    \[
      \frac{d}{d x}\left( x^2 + y(x)^2 \right) = \frac{d}{d x} 0 = 0.
    \]
    This gives
    \[
      2x + 2 y(x) \frac{d y(x)}{dx} = 0
    \]
    or
    \[
      \frac{dy}{dx} = -\frac{2x}{2y}
    \]
    Plug in $x=3$, $y=-4$ to find $\frac{dy}{dx} = 3/4$.

    This second method is known as the \textbf{implicit differentiation}.
  \end{solution}

  \begin{example}
    Find an equation of the tangent line to the curve $x \sin(xy - y^2) = 0$ at $(1, 1)$
  \end{example}

  \begin{example}
    Find $y''$ in terms of $x$ and $y$ if $x y + y^2 = 2x$.
  \end{example}

  \subsection*{The General Power Rule for Derivative}
  So far, we proved the following rule
  \[
    \frac{d x^r}{dx} = r x^{r-1}
  \]
  for integer exponents $r$ and a few special exponents such as $r=1/2$. Using the implicit differentiation, we can give a proof for any rational exponent $r=m/n$ where $n\neq0$.

  If $y=x^{m/n}$ then $y^n = x^m$. Differentiating implicitly
  \[
    n y^{n-1} \frac{dy}{dx} = \frac{d y^n}{dx} = \frac{dx^m}{dx} = m x^{m-1}
  \]
  Hence
  \[
    \frac{dy}{dx} = \frac{m}{n} x^{m-1} y^{1-n} = r x^{m-1} x^{(1-n) m/n} = r x^{m-1 + r -m} = r x^r.
  \]
\end{document}
