\documentclass[../calc1-main.tex]{subfiles}

\begin{document}

To evaluate the limit $\lim_{x \to 0} \frac{\sin x}{x}$ we can not plug in $x=0$.
We call $\sin x/ x$ an \textbf{indeterminate form} of $[0/0]$ at $x=0$.

The limit of an indeterminate form $[0/0]$ can be any number.
\[
  \lim_{x \to 0} \frac{x}{x} = 1, \qquad
  \lim_{x \to 0} \frac{x}{x^3} = \infty, \qquad
  \lim_{x \to 0} \frac{x^3}{x^2} = 0.
\]

There are other types of indeterminate forms $[\infty/\infty]$, $[0 \cdot \infty]$, $[\infty - \infty]$, $[0^{\infty}]$, $[\infty^0]$, $[1^{\infty}]$.


\begin{theorem}[l'Hopital's Rules]
  Let $f$ and $g$ are differentiable on an interval containing $a$.
  Suppose that
  $\lim_{x \to a} f(x)$ and $\lim_{x \to a} g(x)$ are either both 0 or both $\pm \infty$. If $\lim_{x \to a} \frac{f'(x)}{g'(x)}$ exists then
  \[
    \lim_{x \to a} \frac{f(x)}{g(x)} = \lim_{x \to a} \frac{f'(x)}{g'(x)}.
  \]
  The results hold true if $\lim_{x \to a}$ is replaced by $\lim_{x \to a+}$ and $\lim_{x \to a-}$ or if $a= \pm \infty$.
\end{theorem}
\begin{proof}
  Proof follows from generalized mean value theorem which we did not cover.

  Let's give a proof for the following special case. Suppose $f(a) = g(a) = 0$, $g'(a) \neq 0$ and $f$, $g$ have continuous derivatives at $x=a$. Then
\[
  \lim_{x \to a} \frac{f(x)}{g(x)} = \frac{\lim_{x \to a} \frac{f(x)-f(a)}{x-a}}{\lim_{x \to a} \frac{g(x)-g(a)}{x-a}} = \frac{f'(a)}{g'(a)} = \lim_{x \to a} \frac{f'(x)}{g'(x)}
\]
\end{proof}

\texttt{Note that in applying l'Hopital's rule we calculate the quotient of the derivatives, not the derivative of the quotients.}

\begin{example}
  Evaluate
  \[
    \lim_{x \to 1} \frac{\ln x}{x^2-1}
  \]
\end{example}
\begin{solution}
  \[
    \lim_{x \to 1} \frac{\ln x}{x^2-1} \quad \left[ \frac{0}{0} \right]
  \]
  \[
    \lim_{x \to 1} \frac{\ln x}{x^2-1} =
    \lim_{x \to 1} \frac{\frac{1}{x}}{2x} =
    \lim_{x \to 1} \frac{1}{2x^2} = \frac{1}{2}.
  \]
\end{solution}

If one application of the l'Hopital's rule again gives an indeterminate form, we can apply it again.

\begin{example}
  Evaluate
  \[
    \lim_{x \to 0} \frac{2\sin x-\sin(2x)}{2e^x-2-2x-x^2}
  \]
\end{example}
\begin{solution}
  Applying l'Hopital's rule three times we get the answer $3$.
\end{solution}

\begin{example}
  \[
    \lim_{x \to 1+} \frac{x}{\ln x}
  \]
\end{example}
\begin{solution}
  If you apply the l'Hopital's rule, you get the wrong answer of 1. This is not an indeterminate form, and you can't use l'Hopital's rule. The real answer is $\infty$.
\end{solution}

\begin{example}
  \[
    \lim_{x \to 0+} \frac{1}{x} - \frac{1}{\sin x}
  \]
\end{example}
\begin{solution}
  This is an indeterminate form of type $[ \infty - \infty]$ which can be brought to the form $[0/0]$.
  \[
    \lim_{x \to 0+} \frac{1}{x} - \frac{1}{\sin x} =
    \lim_{x \to 0+} \frac{\sin x - x}{x \sin x} =
    \lim_{x \to 0+} \frac{\cos x - 1}{\sin x + x \cos x} =
    \lim_{x \to 0+} \frac{-\sin x}{\cos x + \cos x - x \sin x} = \frac{0}{-2} =0.
  \]
  where we use l'Hopital's rule twice.
\end{solution}

To deal with indeterminate forms of types $[0^0]$, $[\infty^0]$ and $[1^{\infty}]$, we take logarithms.

\begin{example}
  \[
    \lim_{x \to 0+} x^x.
  \]
\end{example}
\begin{solution}
  This is of the form $[0^0]$. Let $y= x^x$. Then
  \[
    \lim_{x \to 0+} \ln y = \lim_{x \to 0+} x \ln x = \lim_{x \to 0+} \frac{\ln x}{1/x}
    = \lim_{x \to 0+} \frac{1/x}{-1/x^2} = 0
  \]
  Since $\ln$ is a continuous function
  \[
    \ln \lim_{x \to 0+} \ln y = \lim_{x \to 0+} \ln y = 0,
  \]
  \[
    \lim_{x \to 0+} x^x = e^0 = 1.
  \]
\end{solution}

\begin{example}
  Evaluate
  \[
    \lim_{x \to \infty} \left( 1+ \sin \frac{3}{x} \right)^x
  \]
\end{example}
\begin{solution}
  This is of the form $[1^{\infty}]$. Again first evaluate the limit of the logarithm. $y=\left( 1+ \sin \frac{3}{x} \right)^x$.
  \[
    \lim_{x \to \infty} \ln y =
    \lim_{x \to \infty} \frac{\ln \left( 1+ \sin \frac{3}{x} \right)}{1/x} = 3
  \]
  Hence
  \[
    \lim_{x \to \infty} \left( 1+ \sin \frac{3}{x} \right)^x = e^3.
  \]
\end{solution}

\rule{\textwidth}{1pt}
\begin{multicols}{2}
\begin{exercise}
~\\
  \begin{enumerate}
    \item $\lim_{x \to \infty} \frac{x^2}{e^x}.$

    Answer: 0
    \item $\lim_{x \to 0} \dfrac{x-\sin x}{x^3}$

    Answer: $1/6$
    \item $\lim_{x \to 0} \dfrac{x-\sin x}{x-\tan x}$

    Answer: $-\frac{1}{2}$
    \item $\lim_{x \to 0+} x^{\sqrt{x}}$

    Answer: $1$
    \item $\lim_{x \to 1} \frac{\ln (ex) - 1}{\sin \pi x}$

    Answer: $-\frac{1}{\pi}$
    \item $\lim_{x \to 0+} \frac{\csc x}{\ln x}$

    Answer: $\infty$
    \item $\lim_{x \to 0} \left( 1+\tan x \right)^{1/x}$

    Answer: $1$
  \end{enumerate}
\end{exercise}
\end{multicols}
\rule{\textwidth}{1pt}


\end{document}
