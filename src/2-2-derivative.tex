\documentclass[calc1-main.tex]{subfiles}

\begin{document}

\begin{definition}
    The \textbf{derivative} of a function $f$ at $x$ is
    \[
        f'(x) = \lim_{h \to 0} \frac{f(x+h)-f(x)}{h}.
    \]
    whenever the limit exists. If $f'(x)$ exists, $f$ is called \textbf{differentiable} at $x$.
\end{definition}

\begin{center}
    \texttt{$f'(x)$ is the slope of the tangent line to the graph of $f$ at $(x, f(x))$.}
\end{center}

We will regard $f'$ as a function whose domain is those $x$ at which $f$ is differentiable.

Another way of defining derivative is
\[
    f'(x_0) = \lim_{x \to x_0} \frac{f(x)-f(x_0)}{x-x_0} = \lim_{h \to 0} \frac{f(x_0+h)-f(x_0)}{h}
\]
Two limits are equivalent. This can be seen by letting $x=x_0 + h$.

\begin{example}
    Show that the derivative of the linear function $f(x) = ax+b$ is $f'(x) = a$. In particular the derivative of a constant function is zero.
\end{example}

\begin{example}
    Use the definition of the derivative to calculate the derivatives of
    a) $f(x) = x^2$, b) $f(x) = \frac{1}{x}$, c) $f(x) = \sqrt{x}$.
\end{example}

The previous three formulas are special cases of the following \textbf{Power Rule for Derivative}:
\[
    f(x) = x^r \implies f'(x) = r x^{r-1}
\]
whenever $x^{r-1}$ makes sense.

\begin{example}
    \[
       f(x) = x^{5/3} \implies f'(x) = x^{2/3},
    \]
    for all $x$. How about $f'(-1/8)$?
    \[
        f(x) = \frac{1}{\sqrt{x}} \implies f'(x) = -\frac{1}{2} x^{-3/2}
    \]
    for $x>0$.
\end{example}

\begin{example}
    Differentiate the absolute value function $f(x) = \abs{x}$ to get
    \[
        f'(x) = \text{sgn}(x) =
        \begin{cases}
            -1, &\text{ if } x<0\\
            1, &\text{ if } x>0
        \end{cases}
    \]
    Note that $f$ is not differentiable at $0$.
\end{example}

\begin{example}
    How should the function $f(x) = x \text{sgn}(x)$ be defined at $x=0$ so that it is continuous there? Is it then differentiable there?
\end{example}

\subsection*{Notations for Derivative}
Let $y = f(x)$. We denote the derivative by
\[
    y' = f'(x) = \frac{dy}{dx} = \frac{d}{dx} f(x).
\]
If we want to evaluate the derivative at point $x_0$
\[
    y' \mid_{x=x_0} = f'(x_0) = \frac{dy}{dx} \mid_{x=x_0} = \frac{d}{dx} f(x) \mid_{x=x_0}.
\]
The notations $y'$ and $f'(x)$ are \textit{Lagrange notations} for the derivative.
The notations $\dfrac{dy}{dx}$ and $\dfrac{d}{dx} f(x) $ are called \textit{Leibniz notations} for the derivative.

The Leibniz notation is suggested by the definition of the derivative. Let $\Delta y = f(x+h) - f(x)$ be the increment in $y$ and $\Delta x = x+h - x = h$ be the increment in $x$. Then
\[
    \frac{dy}{dx} = \lim_{\Delta x \to 0} \frac{\Delta y}{\Delta x}
\]

% \begin{remark}
%     Although a derivative need not be continuous, it must have the intermediate value property just as a continuous function. Read: Darboux Theorem.
% \end{remark}
\end{document}
