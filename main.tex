\documentclass[12pt]{memoir}
\usepackage{subfiles} % to use subfiles
\usepackage[top=2.5cm, left=1.5cm, right=1.5cm, bottom=2cm]{geometry}
\usepackage{taylanmacros}
\usepackage{fourier}
\usepackage{multicol}
\usepackage{sidecap}
\pgfplotsset{compat=1.12}
\chapterstyle{bianchi}
\usetikzlibrary{shapes.geometric,fit} % used in P-4-functions.tex
\usepgfplotslibrary{fillbetween} %  5-2-ex2.tikz
\title{Calculus I Lecture Notes}
\author{Taylan Şengül}


\begin{document}
\maketitle
\tableofcontents
\chapter{Precalculus}
    \section{Sets}
    \subfile{src/P-0-sets.tex}
    \section{Real Numbers}
    \subfile{src/P-1-realnumbers.tex}
    \section{Cartesian Coordinates}
    \subfile{src/P-2-cartesian.tex}
    \section{Quadratic Equations}
    \subfile{src/P-3-quadratics.tex}
    \section{Functions and Their Graphs}
    \subfile{src/P-4-functions.tex}
    \section{Operations on Functions}
    \subfile{src/P-5-operfunctions.tex}
    \section{Polynomials and Rational Functions}
    \subfile{src/P-6-polys.tex}


\chapter{Limits and Continuity}
    \section{Informal definition of limits}
    \subfile{src/1-1-limit.tex}
    \section{Limits at Infinity and Infinite Limits}
    \subfile{src/1-2-infiniteLimits.tex}
    \section{Continuity}
    \subfile{src/1-3-continuity.tex}
    \section{Formal definition of Limit}
    \subfile{src/1-4-formal.tex}
    \section{Review Problems}
    \subfile{src/1-5-limit-review.tex}
%     \section{One Sided Limits and Limits at Infinity}
%     \subfile{src/1-6-limit-variations}
%     \section{Properties of limits}
%     \subfile{src/1-2-properties-limits.tex}


\chapter{Differentiation}
    \section{Tangent Lines and Their Slopes}
    \subfile{src/2-1-tangent}
    \section{Derivative}
    \subfile{src/2-2-derivative}
    \section{Differentiation Rules}
    \subfile{src/2-3-diffrules}
    \section{Chain Rule}
    \subfile{src/2-4-chain}
    \section{Derivatives of Trigonometric Functions}
    \subfile{src/2-5-trig}
    \section{Higher Order Derivatives}
    \subfile{src/2-6-higher}
    \section{Mean Value Theorem}
    \subfile{src/2-8-mean}
    \section{Implicit Differentiation}
    \subfile{src/2-9-implicit}
    \section{Exam 1 Review}
    \subfile{src/exam1}
    % \section{Antiderivatives}
    % \subfile{src/2-10-antider}

\chapter{Transcendental Functions}
    \section{Inverse Functions}
    \subfile{src/3-1-inverse}
    \section{Exponential and Logarithmic Functions}
    \subfile{src/3-2-exp}
    \section{The Natural Logarithm and Exponential}
    \subfile{src/3-3-naturallog}
    \section{The Inverse Trigonometric Functions}
    \subfile{src/3-4-inversetrig}

\chapter{Applications of Derivatives}
    \section{Related Rates}
    \subfile{src/4-1-related.tex}
    \section{Indeterminate Forms}
    \subfile{src/4-2-indeterminate-forms.tex}
    \section{Extreme Values}
    \subfile{src/4-3-extreme.tex}
    \section{Concavity and Inflections}
    \subfile{src/4-4-concavity.tex}
    \section{Graphs of Functions}
    \subfile{src/4-5-graph.tex}
    \section{Extreme Value Problems}
    \subfile{src/4-6-extreme-problems.tex}
    \section{Linear Approximation}
    \subfile{src/4-7-linear-approximation.tex}
    \section{Exam 2 Review}
    \subfile{src/exam2}


\chapter{Integration}
    \section{The Definite Integral}
    \subfile{src/5-1-definite-integral.tex}
    \section{The Fundamental Theorem of Calculus}
    \subfile{src/5-2-FTC.tex}
    \section{The Method of Substitution}
    \subfile{src/5-3-substitution.tex}
    \section{Areas of Plane Regions}
    \subfile{src/5-4-areas.tex}
    \section{Integration by Parts}
    \subfile{src/5-5.intByParts.tex}
    \section{Integrals of Rational Function}
    \subfile{src/5-6-integralOfRationalFuncs.tex}
    \section{Inverse Substitutions}
    \subfile{src/5-7-inverseSubstitutions.tex}
\end{document}
